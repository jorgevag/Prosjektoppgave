%%%%%%%%%%%%%%%%%%%%%%%%%%%%%%%%%%%%%%%%%
% Short Sectioned Assignment
% LaTeX Template
% Version 1.0 (5/5/12)
%
% This template has been downloaded from:
% http://www.LaTeXTemplates.com
%
% Original author:
% Frits Wenneker (http://www.howtotex.com)
%
% License:
% CC BY-NC-SA 3.0 (http://creativecommons.org/licenses/by-nc-sa/3.0/)
%
%%%%%%%%%%%%%%%%%%%%%%%%%%%%%%%%%%%%%%%%%

%----------------------------------------------------------------------------------------
%	PACKAGES AND OTHER DOCUMENT CONFIGURATIONS
%----------------------------------------------------------------------------------------

\documentclass[norsk]{article} % A4 paper and 11pt font size

\usepackage[T1]{fontenc} % Use 8-bit encoding that has 256 glyphs
\usepackage{fourier} % Use the Adobe Utopia font for the document - comment this line to return to the LaTeX default
\usepackage[english]{babel} % English language/hyphenation
\usepackage{amsmath,amsfonts,amsthm} % Math packages

% Added by Joergen %---------------------------------------------------------------------
%\usepackage{cleveref}

% Added by Haavard %---------------------------------------------------------------------
\usepackage[utf8]{inputenc} % Norwegian letters
\usepackage{fullpage}
\usepackage{subcaption}
\usepackage[font={small, it}]{caption} % captions on figures and tables
\usepackage{graphicx}
\usepackage{color}
\usepackage{hyperref} % Use \autoref{ and \nameref{
\hypersetup{backref,
  colorlinks=true,
  breaklinks=true,
  %hidelinks, %uncomment to make links black
  linkcolor=blue,
  urlcolor=blue,
  citecolor=blue
}
\usepackage[all]{hypcap} % Makes hyperref jup to top of pictures and tables
% --------------------------------------------------------------------------------------

\usepackage{lipsum} % Used for inserting dummy 'Lorem ipsum' text into the template

\usepackage{sectsty} % Allows customizing section commands
\allsectionsfont{\centering \normalfont\scshape} % Make all sections centered, the default font and small caps

\usepackage{fancyhdr} % Custom headers and footers
\pagestyle{fancyplain} % Makes all pages in the document conform to the custom headers and footers
\fancyhead{} % No page header - if you want one, create it in the same way as the footers below
\fancyfoot[L]{} % Empty left footer
\fancyfoot[C]{} % Empty center footer
\fancyfoot[R]{\thepage} % Page numbering for right footer
\renewcommand{\headrulewidth}{0pt} % Remove header underlines
\renewcommand{\footrulewidth}{0pt} % Remove footer underlines
\setlength{\headheight}{13.6pt} % Customize the height of the header

\numberwithin{equation}{section} % Number equations within sections (i.e. 1.1, 1.2, 2.1, 2.2 instead of 1, 2, 3, 4)
\numberwithin{figure}{section} % Number figures within sections (i.e. 1.1, 1.2, 2.1, 2.2 instead of 1, 2, 3, 4)
\numberwithin{table}{section} % Number tables within sections (i.e. 1.1, 1.2, 2.1, 2.2 instead of 1, 2, 3, 4)

\setlength\parindent{0pt} % Removes all indentation from paragraphs - comment this line for an assignment with lots of text

%----------------------------------------------------------------------------------------
%	TITLE SECTION
%----------------------------------------------------------------------------------------

\newcommand{\horrule}[1]{\rule{\linewidth}{#1}} % Create horizontal rule command with 1 argument of height

\title{	
\normalfont \normalsize 
\textsc{NTNU 2015} \\ [25pt] % Your university, school and/or department name(s)
\horrule{0.5pt} \\[0.4cm] % Thin top horizontal rule
\huge Notes \\ Engauge Digitizer \\ % The assignment title
\horrule{2pt} \\[0.5cm] % Thick bottom horizontal rule
}

\author{Jørgen Vågan} % Your name

\date{\normalsize\today} % Today's date or a custom date


\begin{document}


\maketitle % Print the title

%----------------------------------------------------------------------------------------
%	Text Body:
%----------------------------------------------------------------------------------------

%\listoftodos{}

\abstract{These notes are mainly taken from:\\
http://digitizer.sourceforge.net/usermanual/index.html\\
\\
Engauge Digitizer is a software to digitize graphed data from text books and 
published papers, which means that it extracts data from graphs/images. \\
Just import an image, define three coordinate points and their scales,
pick off the points and export the digitized data to a file.\\
\\
Key features:
\begin{itemize}
\item The definition of 3 axes allows the software to correct for rotation or uniform 
skrew in the graph.
\item You can specify whether the data is in linear or log scales.
\item You can define multiple ''curves'' to help segregate data.
\end{itemize}
}

\newpage
\section{Installation}
Command line: \\
\texttt{ sudo apt-get install engauge-digitizer} \\

\section{General Steps}
\textbf{Typical Steps:}
\begin{enumerate}
   \item Obtain image file (bmp, jpeg or other) showing one or more curves and both axes;
   \item Import image file using either:
      \begin{itemize}
         \item File/Import menu option;
         \item Copy-Paste;
         \item Drag and drop;
      \end{itemize}
   \item If important parts of the image is missing (e.g. parts of- or entire curves) or 
      curves are too thick (difficult to separate them),
      then go to \texttt{Settings/Discretize} and experiment with the discretization options
      until all curves are there and looking as nice as possible.
   \item Click on \texttt{Axes Point} button. 
   \item Click on click on one of the axes and enter graph coordinates.\\
         Repeat until you've added all axes;


      \item Digitize graph according to subsections below.


   \item Click on the \texttt{Curve Points} button to manually enter curve points by 
      clicking on the curve. Repeat until the curve is covered with a sufficient number of curve points.
   \item Export curve points by using either: 
      \begin{itemize}
         \item File/Export menu option to save selected curves into a tabular text file, or
         \item copy-pase / drag-drop points in the current curve from mthe \texttt{curve geometry window}
               to another application.
         \item Copy-Paste;
         \item Drag and drop;
      \end{itemize}
\end{enumerate}

\subsection{Digitize Line Graphs}
\begin{enumerate}
   \item Click on the \texttt{Segment Fill} button to automatically digitize entire curve
         segments at a time. Click on a segment underneath the cursor to fill that segment with a set
      of curve points. Repeat until the segment have been digitized.
\end{enumerate}

\subsection{Digitize Point Graphs}
\begin{enumerate}
   \item Click on the \texttt{Point Match} button to automatically digitize many curve points.
   \item Click on a sample point and use the arrow keys to accept or reject points that match
      the sample point.
      

\end{enumerate}

\section{Optional Steps}
To clean up image for better eprformance of Segment Fill and Point Match:
\begin{enumerate}
   \item Show processed image rather than original image, by selecting the 
      \texttt{View}/\texttt{Processed Image} menu option
   \item Use \texttt{Settings/Discretize} menu option to adjust 
      and remove unwanted parts of image.
   \item Remove grid lines using \texttt{Settings/Grid Removal} menu option.
\end{enumerate}
For logarithmic or polar graph:
\begin{enumerate}
   \item Select \texttt{Settings/Coords} menu option to bring up the \textit{coordinates window}
      and select the appropriate coordinate settings.
\end{enumerate}
For a graph with several curves
\begin{enumerate}
   \item Select \texttt{Settings/Curves} menu option to bring up the \textit{coordinates window}.
   \item Click on the \texttt{New} button to create a new curve and enter its name.
   \item Reapeat until all curve nemes have been entered.
   \item In the main window, use the Curves combobox to select a curve. While selected, all
      new curve points will be assigned to that curve.
\end{enumerate}

      
\section{Miscellaneous}
For \textbf{Measuring Angles, Distances and Areas} you need to bring up the curve and measure geometry window.
You can bring it up through the \texttt{View/Cuvrve Geometry Info} menu option.\\ 
\\
\textbf{Better Accuracy}. The points are only as accurate as the pixel size (you can see that
accuracy in the sattus bar). If the distance in graph units from one piel to the next is
$D$, then theoretically the best possible accuracy is also $D$.\\
It is not possible to define points that are ''between'' pixels, even by zooming in closer.
That is a constraint imposed by the graphcs library that is used in \textit{Engauge Digitizer}.\\
\\
Fortunately, there are some tricks to improve the accuracy. Here, in order of increasing difficulty:
\begin{enumerate}
   \item If the axis points are not correctly defined then fix them.
   \item Turn on the grid lines by selecting \texttt{View/Gridlines Display} and compare them to
      grid lines in the original image, to see if the axis points can be slightly adjusted so
      the grid lines better match each other.
   \item Use a graphics application like Microsoft Paint(windows) or GIMP (Linux) to scale up the 
      original in size, such that each pixel is smaller, giving better accurary. The larger 
      image is them imported (can be perfomred using the command line and the powerful ImageMagick tool).
\end{enumerate}


%
\end{document}
