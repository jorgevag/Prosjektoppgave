\section{Oppsummering Møte 1.Mai}

\subsection{Mål med oppgaven}
Ha termochromst materiale som coating, enten som granular layer eller som coating utenpå et metallisk 
granular layer. Dette skal kunne ''smøres'' på en byggning/vinduer/el. slik at f.eks varmestrålingen 
slippes inn om vinteren når det er kalt ute og reflekteres om sommeren når det er varmt ute.
\\
\\
Om denne optiske endringen kan oppnår ved temperaturgradienten for sommer/vinter kan man unngå bruk av aktive
løsninger (e.g. bruke spenning for å oppnå tilsvarende optiske egenskaper). Materialene burde også
være billig, slik at dette er realiserbart for storskala-prosjekter som store byggninger.
%
\begin{table}[htbp]
   \caption{oppbyggningen av ''.nk'' filene i Sopra-Databasen}
\centering
\begin{tabular}{ c c c c c }
\hline
 \multicolumn{2}{c}{unit}         &  startverdi  &  sluttverdi & antall datapunkter \\
\hline
1          & 2                    &                 &                  & \\
energi[eV] & bølgelengde[$\mu$m]  &  x1[eV/$\mu$m]  &    x2[eV/$\mu$m] & N \\
\hline
\end{tabular}
\label{tab:idealTCW}
\end{table}
%

Så ''.nk-files'' vil være på følgende form:
\begin{lstlisting}[style=FormattedNumber, language=python]
unit		x1			x2	   N	
     n(x1)          k(x1) 
     n(x2)          k(x2) 
     n(x3)          k(x3) 
     n(x4)          k(x4) 
     n(x5)          k(x5) 
     n(x6)          k(x6) 
     n(x7)          k(x7) 
     n(x8)          k(x8) 
     n(x9)          k(x9) 
     n(x10)         k(x10) 
     n(x11)         k(x11) 
     .              .
     .              .
     .              .
\end{lstlisting}
Eksempel: ''mgo.nk'' i eV, startverdi: $0.65$eV, sluttverdi: $10$eV og består av totalt
400 datapunkter:
\begin{lstlisting}[style=FormattedNumber, language=python]
1		0.65			10		400
     1.70969699     0.00000011
     1.71065583     0.00000011
     1.71155117     0.00000011
     1.71239613     0.00000011
     1.71313635     0.00000011
     .              .
     .              .
     .              .
\end{lstlisting}


\subsection{Videre arbeid med oppgaven}
\begin{itemize}
\item Jeg skal finne data på thermochrome dielectrisitetskonstanter og laga en database basert på dette.
   Dette skal bli matet inn i \textsc{GranFilm}.
\item Output burde kanskje være konturplot (for å lettere lese av verdiene, ihvertfall sammenlignet med 3D-plot)
\item Les artikler og finn ''teori'' for å beskrive $\varepsilon(\omega, T)$, men hovedsakelig for å
finne data for $\varepsilon(\omega, T)$.
\end{itemize}
