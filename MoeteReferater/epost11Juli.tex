\section{Oppsumering E-post 11. Juli}
\begin{itemize}
\item
Anngående dielektriske data: min forståelse av dataen i .nk-fila er riktig.
MEN(!), dataen må være ekvidistansert og helst som kompleks brytningsindex $\hat{n}=n+i\kappa$.
HUSK OGSÅ(!), at n,k-verdier i hver linje i ''.nk'' filen svarer til samme x-verdi.
Jeg må også være oppmerksom på at det finnes to konvensjoner for
tidsavhenigheten til kompleks brytningsindex:
\begin{align}
   e^{-i\omega t}  \:\:\:\:\:\:\:\:\: \text{ og }  \:\:\:\:\:\:\:\:\: e^{+i\omega t} 
\end{align}
Vi bruker førstenevnte (ift. \textsc{GranFilm}), som betyr at positiv
imaginærdel til epsilon betyr absorpsjon ($\Im\{\varepsilon\}>0$). 
(\textit{?er dette påkverd for at det skal være fysisk, type: ingen materialer avgir energy?}).
Jeg må derfor passe på at dataen jeg konverterer har riktig fortegnskonvensjon (elektroingeniører
bruker typisk motsatt av oss).

\item
Kompleks brytningsindeks $\hat{n}=n+i\kappa$ for et ikke magnetisk materiale er definert via
\begin{align}
   \hat{n} = \sqrt{\hat{\varepsilon}(\omega)}, \:\:\:\:\: \text{with Im}\{\hat{n}\} = \kappa > 0,
\end{align}
med vår fortegnskonvensjon.

\item I oppgaven burde jeg klare å lese inn dielektriske data for noen termokromme materialer
   for ulike temperatuerer, samt regne ut hvordan refleksjons egenskapene endrer seg med 
   temperatur i det interessante området ved hjelp av \textsc{GranFilm}.
   Jeg må også diskutere hva disse resultatene betyr.
   \\
   \\
   I tillegg til de eksperimentelle dataene, kan det muligens være av interesse å
   bruke modeller for $\varepsilon(\omega, T)$ om jeg klarer å finne en slik model.
\end{itemize}
