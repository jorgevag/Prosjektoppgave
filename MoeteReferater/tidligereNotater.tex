\section{Tidligere Notater}
To unzip a ''.tgz''-file in linux:
\begin{lstlisting}[style=FormattedNumber, frame=none,language=bash]
   tar xvfz path/file.tgz
\end{lstlisting}
%
For å lese linux filer:
\begin{lstlisting}[style=FormattedNumber, frame=none,language=bash]
   less text.txt
\end{lstlisting}
(q for quit/å avslutte). 
%
For å lage en peker til en executable i en annen mappe (skapt til bruk i hvilken som helst annen mappe), skriv
\begin{lstlisting}[style=FormattedNumber,frame=none, language=bash]
   ln -s RelativePath/yourExecutable .
\end{lstlisting}
For eksempel med \textsc{GranFilm}:
\begin{lstlisting}[style=FormattedNumber,frame=none, language=bash]
   ln -s ../Src/GranFilm .
\end{lstlisting}
For å kjøre \textsc{GranFilm}:
\begin{lstlisting}[style=FormattedNumber,frame=none, language=bash]
   ./GranFilm -p inputParameters.sif -o outputFile.dat
\end{lstlisting}
%
A simple easy plotting program is \textsc{xmgrace}. I think this assumes
data with x-axis at column 1 and everything else on the other columns. Run by typing:
\begin{lstlisting}[style=FormattedNumber,frame=none, language=bash]
   xmgrace dataFile.dat
\end{lstlisting}
or (if you want to plot all the data in the file)
\begin{lstlisting}[style=FormattedNumber,frame=none, language=bash]
   xmgrace -nxy dataFile.dat
\end{lstlisting}

