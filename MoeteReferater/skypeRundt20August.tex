\section{Fra samtaler $\sim$ 20. August}

\begin{itemize}
   \item 
   Lyset burde ligge i området 150nm - 200nm i følge Ingve.\\
   Dataen min fra Kang et al. 2012 ligger i området 0,761-3,987 eV, tilsvarende 311.0-1629.2nm.

   \item Med den gitte bølgelengden burde radiusen kanskje ligge på omtrent $\sim$ 10-15nm.
      (Pga. quasi-static approximation: $\lambda \gg r$) 
      \\
      Basert på det jeg ved selv:\\
      Basert på det over virker det som om partikkel radiusen
      ikke burde overgå ca. 10\% av størrelsen til bølgelengden. 
      Siden dataen min ligger i området 311.0-1629.2nm vil dette kunne tillate meg partikkelstørrelser
      opp til 30nm. Å ligge godt innenfor området vil trolig være lurt uansett, siden 1\%-5\% av bølgelengden
      vil være intervallet rundt 3-15nm. Dette samsvarer også med intervallet han har gitt meg.


   \item  avstanden mellom sentrum til sentrum (lattice constant $a$) burde være større en cirka
      2,5-3 ganger radiusen til partiklene, i.e.
      \begin{align}
         a > 3R, \:\:\:\:\:\:\: \text{(tommelfingerregel)}
      \end{align}
      for å unngå for kraftig interaksjon mellom partiklene (?og det er kunn blitt tatt hensyn til dipol
      interaksjon mellom partiklene, type pertubasjon, så størrelsen/pertubasjonen pga de andre
      partiklene burde ikke være for stor?).

    \item
       $\Delta R/R$ gir den totale refleksjonen fra alle effekter. Å se på polarisibiliteten
       kan hjelpe å se hvordan resonnansene kommer fra effekter som er forholdsvis vinkelrett eller
       parallel med overflaten (vektorstørrelse) og gir mer detaljert informasjon enn 
       refleksjonen alene.
\end{itemize}
