\section{Spørsmål; Runde 3}
\begin{itemize}
   \item \textbf{Plasmoner og Overflate-Plasmoner}
      \begin{enumerate}
         \item Er størrelsene for ''polarizabilities'' $\alpha_{\parallel}$ og $\alpha_{\perp}$
            direkte relatert til plasmoner?
         \item Blir både bulk- og overflate-plasmoner eksitert i/på den granulære overflaten?
         \item Hvis ja, hvordan avviker den granulære overflaten fra en flat, glatt overflate
            mtp. plasmoner. \\
            Leste noe om at ujevnheter i overflaten ville føre til at overflate-plasmoner
            vil sende ut EM-stråling. Vil noe slikt føre til økt refleksjon av den inkommende bølgen?
      \end{enumerate}

   \item \textbf{"The coice of the reference Fresnel surface"/ "choice of the separation surface"}\\
      Dette blir nevnt under diskusjonen av "surface susceptibilities": $\delta, \tau, \gamma \beta$
      i GranFilm-artikkelen til deg og Lazzari på s.126 (siste avsnitt).
      Det står også at Fresnel størrelsene vil ikke endre seg mtp. valg av Fresnel overflate.
      Jeg forstår det slik at dette bare dreier seg om valg av referansesystem, altså 
      om man velger at den fysiske overflaten er satt $z=0$ eller en eller annen 
      $z=z_1 \neq 0$. Er dette riktig?
\end{itemize}

