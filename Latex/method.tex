\section{Method}
%\begin{enumerate}
%\item Searched articles for thermochromic dielectric function or refractive index as a function of 
         %frequency/energy and temperature.

%\item Used \textsc{Engauge Digitizer} to extract the data from the article plots.

%\item The resulting data was not equidistant nor given for the same energy values and could be
%either complex refractive index or complex dielectri constant. \textsc{GranFilm} requires
%equidistanced data with both a real and imaginary value for each energy/wavelength. This had to be solved.

%\item Wrote a code converting the extracted data to a '.nk'-input files which could be read by \textsc{GranFilm}.
%The was done by interpolating the data from the different files, such extracting common and equidistant 
%x-values for both the real and imaginary parts. If the data was given as a dielectric constant this was 
%converted to the refracive index.
%\\ 
%\\
%The code was written as generic as possible to tackle different units such as energy in eV ans wavelength  
%in m, $\mu$m, nm. Should also have included tackling frequency in Hz.

%\item The data from the different sources were stored in a database with one 'nk'-file for each
%temperature. These files were then fed into \textsc{GranFilm}.

%\item The data chosen to be simulated was from REFERENCE KANG 2012. To visualize the temperature and 
   %energydependence of the dielectric constant, the data was interpolated in both energy and temperature.
%\end{enumerate}
%
By searching through various scientific articles, thermochromic data for a material in the 
form of either a complex dielectric constant $\widehat\varepsilon (\omega,T)$ or complex refrective index 
$N(\omega, T)$ was found. By the use of a program called \textsc{Engauge Digitizer} graphically represented
data, such as graphs or plots, could be extracted for use in numerical calculations. \textsc{Engauge Digitizer}
is a open source digitizing software and can be found on
http://digitizer.sourceforge.net/.
It can convert image files of type \texttt{.bmp, .jpeg} or other,
containing a graph or a map, into numbers. The conversion from graphical to numerical data
is obtained through defining three 
points the image, e.g. the origin, largest x-value and largest y-value for a function y = f(x),
which define the x,y-axes together with their scale. 
The target data can be marked automatically or manually, depending on the quality of the image. 
The data can then be exported to a spreadsheet and used in numerical calculation.
\\
\\
The material data input to be read by \textsc{GranFilm} must however be on a specific form. 
%\textsc{GranFilm} requires a complex refractive index $N$ with both the real and imaginary values 
%given for each value of a certain energy or wavelength. The values must be given for equidistant
%values of energy and wavelength
\textsc{GranFilm} requires a complex refractive index $N$ as a discrete function of energy or wavelength. 
The real and imaginary values of $N$ must be given for each point of its domain. In addition, the 
values of the domain must consist of equdistant values. The resulting data through the use of 
\textsc{Engauge Digitizer} is not equidistant and because the real and imaginary data is usually
extracted from different image files, their values do not correspond to the same discrete domain. 
\\
\\
Because of this mismatch between the extracted image data and the input file format of \textsc{GranFilm},
some conversion had to be done. A generic conversion program was made,
set to handle data of either $N$ or $\hat\varepsilon$ in some expected units,
such as $\widehat\varepsilon(E,T)$, $N(E,T)$, 
with [$E$]=eV, and $\widehat\varepsilon(\lambda,T)$,$N(E,T)$, with [$\lambda$] $\in$ \{m,$\mu$m,nm\}. 
It was also encountered data as a function of frequency in Hz. This situation was not implemented, 
but should have been included if further work were to be made. 
\\
\\
The converted data were then stored in a database with one file representing the spectral 
optical index, $N(E,T_i)$, $N(\lambda,T_i)$, for a spesific temperature $T_i$. The data could then 
be fed into \textsc{GranFilm} one file at the time and simulated. 
\\
\\
Due to lacking overlap between the already included material data in \textsc{GranFilm} and the new extracted
data, only data from Kang et al. \cite[p.~3]{Kang2012}, containing the dielectric constant of 
VO$_2$ have been used. 
Based on the produced code, it would not be a lot of work extracting more data 
to expand the thermochromic database, maybe including several thermochromic materials. However, the
extraction of the data with \textsc{Engauge Digitizer} dependens on the quality and 
separation of the functions and could in the worst cases be very tedious.

\subsection{Choice of parameters}
Due to the quasistatic approximation where it was assumed that the layer thickness or the size
of the spheres should be negligible compared to the wavelength of the incident light $d,r \ll \lambda$.
\begin{itemize}
   \item talk about the corresponding intervall we're simulating (the energy range and the corresponding
      wavelength range).
   \item Then talk about how large sphere sizes we're then allowed to use.
   \item relate this up towards the mentioned thicness mentioned by the sources and comment on it.
\end{itemize}
(...)can not use the thin optimal film layer thickness between 40-90nm as mentioned by Kamalisarvestani et al. \cite{Kamalisarvestani2013} and Blackman 
et al. \cite{Blackman2009}.


\begin{thebibliography}{9}

   \bibitem{Kang2012}
   Kang M, Kim SW, Ryu JW, Noh T.
   Optical properties for the Mott transition in VO2.
   AIP Advances 2012;2,012168 \textbf{Er dette greit? Finner ikke sidetall osv.}
\end{thebibliography}


