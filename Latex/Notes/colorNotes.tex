\section{Color notes from wikipedia}
\begin{itemize}
\item Humans tend to percieve light within the green parts of the spectrum as brighter than
red or blue light of equal power. The luminosity function that describes the percieved brightnesses
of different wavelengths is thus roughly anagagous to the spectral sensity of M cones.
\item \textbf{With three color apperance parameters- colorfulness(or chroma or saturation), lightness 
   (or brightness), and hue, any color can be described.}
\item \textbf{Hue} is one of the main properties of a color. There are six (unique) hues: 
   red, orange, yellow, green, blue and purple. ?Hue is described as the degree to which
   a stimulus can be described as similar to or different from stimuli that are described by the unique
   hues?
\item \textbf{Colorfulness} is the visual sensation according to which the percieved color of an area appears
   to be more or less chromatic. A highly colorful stimulus is vivid and intense, while a less colorful
   stimulus appears more muted, closer to gray. With no colorfulness at all, a color is a
   "neutral" gray (an image with no colorfulness in any of its color is called grayscale).
\item \textbf{Chroma} is the colorfulness relative to the brightness of a similarly illuminated area that 
   appears to be white or highly transmitting. (therefore) Chroma should not be confused with colorfulness!
\item Saturation is the colorfulness of a color relative to its own brightness.
\item \textbf{Lightness} (aka value or tone) is a representation of variation in the perception of
   a color or color space's brightness. Lightness is a relative term. Lightness means Brightness
   of an area judged relative to the brightness of a similarly illuminated area that appears to be white 
   or highly transmitting. Lightness should not be confused with brightness.
\item \textbf{Brightness} is an attribute of visual perception in which a source appears
   to be radiating or reflecting light. Brightness is the perception elicited y the luminance 
   (how much luminous power will be detected by an eye looking at e.g. a surface from a particluar angle
   of view) of a visual target. This is a sibjective property of an object being observed and 
   one of the color apperance parameters of color apperance models. Brightness refers to an
   absolute term and should not be confused with Lightness.
   \\
   \\
   In RGB color space, brightness can be thought of as the arithmetic mean $\mu$ of the red, green and blue
   color coordinates (although some of the three components make the light seem brighter than others).
   \begin{align}
      \mu = \frac{R + G + B}{3}
   \end{align}
   Brightness is also a color coordinate in the HSB or HSV color space (Hue, Saturation, and Brightness/Value).
\end{itemize}

\newpage
\section{Color notes regarding \textsc{ColorPy}}
source: http://markkness.net/colorpy/ColorPy.html
\\
\\
How I installed it:
\begin{itemize}
   \item download zipped folder and unzip it using:\\
      gunzip -c colorpy-0.1.0.tar.gz | tar xf
      \\
      \\
      The unzipped folder can be put anywhere (doesn't matter). Go to it (cd path/colorpy-0.1.0).
      Personally, I ran the following command (don't know if you need administrator privileges, 
      but it seemed like it was required: \\
      sudo python setup.py install 
      \\
      \\
      After than I made a folder ''TEST'', went into it, started python and ran the following commands to
      test that the program/module worked correctly:\\
      \# running test cases: \\
      import colorpy.test\\
      colorpy.test.test()\\
      \\
      \# generate sample figures: \\
      import colopy.figures\\
      colorpy.figures.figures()\\
      \\
\end{itemize}
Fundamentals:
\begin{itemize}
   \item \textsc{ColorPy} generally uses wavelengths measured in nanometers, otherwise typical metric
      units are used.
   \item The description of the spectra, \textsc{ColorPy} uses two-dimensional NumPy arrays:
      column 1: wavelength[nm], column 2: light intensity at the wavelength.
   \item Color values are represented as three-component NumPy vectors (1D-arrays). Typically, these are 
      vectors of floats, except for irgb colors, which are arrays of integers in the range 0-255.
   \item \textsc{ColorPy} can provide a blank spectrum array, via colorpy.ciexyx.empty\_spectrum(),
      which have rows for each wavelength from 360nm to 830nm, at 1nm increments.
   \item To go from the light spectra to a color (a 3D quantity), we use a set of three matching functions
      \begin{align}
         X = \int I(\lambda) \cdot C\!I\!E\!-\!\!X(\lambda) d\!\lambda\\
         Y = \int I(\lambda) \cdot C\!I\!E\!-\!\!Y(\lambda) d\!\lambda\\
         Z = \int I(\lambda) \cdot C\!I\!E\!-\!\!Z(\lambda) d\!\lambda
      \end{align}
   \item The Y-matching function corresponds exactly to the luminous efficiency of the eye - the eye's
      response to light of constant luminance (these facts are some of the reasons that make this 
      particular ser of matching functions so useful).
   \item Rescaling XYZ so that their sum is 1.0, we get the values xyz (i.e. $x + y + z = 1$).
      The chromaticity is given by x and y (z can be recontructed as $z = 1.0 - x - y$). 
      It is also common to specify colors with the chromaticity (x,y) , as well as the total
      brightness (Y). Occasionally, one also wants to scale XYZ colors so that the resulting Y value is 1.0.

   \item Here are some 'constructor'-like functions for creating the colors (e.g.XYZ), which are 
      three-compontent vectors.
\begin{lstlisting}[style=FormattedNumber, language=python, frame=none]
   colorpy.colormodels.xyz_color(x,y,z=None)
   colorpy.colormodels.xyz_normalize(xyz)
   colorpy.colormodels.xyz_color_from_xyY(x,y,Y)
   colorpy.colormodels.xyz_normalize_Y1(xyz)
\end{lstlisting}
   Note that color types are generally specified in ColorPy with lower case letters, as this is more 
   readable. The user must keep track of the particular normalization that applies in each situation.
\end{itemize}
Fundamentals - converting XYZ colors to RGB colors which we can draw on the computer:
\begin{itemize}
   \item To convert we can use
\begin{lstlisting}[style=FormattedNumber, language=python, frame=none]
   # Here, xyz is the XYZ color vector:
   colorpy.colormodels.irgb_from_xyz(xyz) # Returns [int,int,int] (int in [0,255])
   colorpy.colormodels.irgb_string_from_xyz(xyz) # Returns hex string, e.g. '#FF0000' for red.
\end{lstlisting}

\item 
Note that there are several subtleties and approximations which are important to understand
what is happening in the code above.
   \begin{itemize}
      \item the XYZ color is converted to a linear RGB color(represented as floats in the range 0.0-1.0),
         meaning that the light intensity is 
         proportional to the numerical color values, and is done by a 3x3 element array. The values
         of the array depends on physical display in question, so the conversion matric cannot apply to 
         all displays/monitors. Fortunately, we can assume a specification of monitor chromaticities
         (which are a part of the sRGB standard) which are likely to be close to most displays. 
         \textsc{ColorPy}
         uses this assumption by default, although you can change the assumed monitor chromaticities to nearly
         anything you like.
      \item The RGB colors can be out of range: 
         \begin{enumerate}
            \item greater than 1.0, meaning the color is too bright for the display. 
            \item negative value, meaning that the color is too saturated and vivid for the display.
         \end{enumerate}
         To put these values between 0.0-1.0 that can actually be displayed is known as color clipping:
         The first is solved by rescaling such that the maximum value is 1.0, making the 
         color less bright without changing the chromaticity. The second change required some change in
         chromaticity. By default,  \textsc{ColorPy} adds just enough white to the color to make all the
         components non-negative. (There is also the potential to develop better clipping functions.)
      \item The next subtlety is that the intensity of colors on the display is not simply
         proportional to the color values given to the hardware. This situation is known as 
         the 'gamma corretion'. We rely on the sRGB standard what to do. The standard assumes a 
         physical 'gamma' exponent of about 2.2., and \textsc{ColorPy} applies this correction by default
         (can be changed if you like).
      \item The final step is to convert the RGB components from 0.0-1.0 to 0-255 (done with simple scaling
         and rounding).
   \end{itemize}
\item Summarizing these conversions:
   \begin{itemize}
      \item The function
\begin{lstlisting}[style=FormattedNumber, language=python, frame=none]
   colorpy.colormodels.rgb_from_xyz(xyz)
\end{lstlisting}
      converts an XYZ color to linear RGB color in range 0.0-1.0. Can be out of range or negative, and
      can not be directly passed to drawing functions.
   \item The function
\begin{lstlisting}[style=FormattedNumber, language=python, frame=none]
   colorpy.colormodels.irgb_from_rgb(rgb)
\end{lstlisting}
      converts a linear RGB in the range 0.0-1.0 to a displayable irgb color, definitely in the range 
      0-255. Color clipping may be applied (intensity and chromaticity), and gamma correction is 
      accounted for. This result can be passed to drawing functions.
   \end{itemize}

\item To plot the visible spectrum:
\begin{lstlisting}[style=FormattedNumber, language=python, frame=none]
   colorpy.plots.visible_spectrum_plot()
\end{lstlisting}
To plot the RGB values for the pure spectral lines (?or visible spectrum?), use:
\begin{lstlisting}[style=FormattedNumber, language=python, frame=none]
   colorpy.plots.color_vs_param_plot(param_list, rgb_colors, title, 
      filename, tight=False, plotfunc=pylab.plot, xlabel='param', ylabel='RGB Color')
\end{lstlisting}
In this case, the parameter list is wavelength and we also include a list of RGB colors. These must be of
the same size. This is a very handy function, useful for many other plots besides this one.

\item CIE chromaticity diagram of the visible gamut:
\begin{lstlisting}[style=FormattedNumber, language=python, frame=none]
   colorpy.plots.shark_fin_plot()
\end{lstlisting}

\item 'Patch' plot:
\begin{lstlisting}[style=FormattedNumber, language=python, frame=none]
   colorpy.plots.xyz_patch_plot(xyz_colors, color_names, 
      title, filename, patch_gap=0.05, num_across=6)
\end{lstlisting}
\texttt{xyz\_colors} and \texttt{color\_names} are two lists (which must be of same size).
You can pass 'None' for the second list to skip the labels. (There are also optional arguments to fine-tune
the size and arrangement of the patches.) \\
There is also a similar function:
\begin{lstlisting}[style=FormattedNumber, language=python, frame=none]
   colorpy.plots.rgb_patch_plot()
\end{lstlisting}
when you have known RGB values that you want to draw.

\item To plot spectrums like black body spectrums, use
\begin{lstlisting}[style=FormattedNumber, language=python, frame=none]
   colopy.plots.spectrum_plot(spectrum,title,filename,xlabel,ylabel)
\end{lstlisting}
Here, \texttt{spectrum} is a numpy array.
\\
\\
\textsc{ColorPy} allows us to plot the resulting color vs. temperature, while also showing a plot
of the rgb color values:
\begin{lstlisting}[style=FormattedNumber, language=python, frame=none]
   colopy.plots.color_vs_param_plot(param_list, rgb_colors, 
      title, filename, tight, plotfunc, xlabel, ylabel)
\end{lstlisting}
The arguments \texttt{tight, plotfunc, xlabel, ylabel} are optional. (In the example, he used
tight=True, plotfunc=pylab.semilogy to obtain the semilog plot, which is needed for the very large range
of color values covered).
\end{itemize}
