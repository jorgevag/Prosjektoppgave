\section{\textbf{Book Notes}}
\section{Handbook of Optial Constants of Solids; Edward D. Palik}
Institute for physical Science and Technology, University of Maryland\\
Academic Press 1998, 1985.\\

\subsection{Bulk and thin-film effects; effective-medium theory; p.104}
\begin{itemize}
\item p. 105:\\
   In this discussion, we assume that the characteristic dimensions of the microstructure
   are large enough ($>/\sim 10-20 $Å) so that the individual regions retain essentially their bulk dielectric
   responses, but small ($>/\sim$ 0.1-0.2 $\lambda$) compared to the wavelength of light.
   Then, the macroscopic $\boldsymbol E$ and $\boldsymbol H$ fields of Maxwell's equations will not vary
   appreciably over any single region, and quasistatic theory can be used. This avoids complications
   due to scattering and retardation effects that are dominant in macroscopiccally inhomogeneous 
   systems \cite{Beckmann1968}.
   \\
   \\
   The dielectric functions is obtained from the macroscopic average electric field $\boldsymbol E$ and 
   polarization $\boldsymbol P$ according to
   \begin{align}
      \boldsymbol D &= \varepsilon \boldsymbol E = \boldsymbol E+ 4\pi \boldsymbol P \\
      \boldsymbol P &= \frac{1}{V} \sum q_i \Delta\! \boldsymbol{x_i},
   \end{align}
   where $\Delta \! \boldsymbol{x_i}$ is the displacement of the charge $q_i$ under the action of the 
   local field at $q_i$. It is the apperance of the volume normalizing factor in the latter equation
   that is responsible for the sensitivity of $\varepsilon$ to density.
   \item 
\end{itemize}

\begin{thebibliography}{9}
      \bibitem{Beckmann1968}
      Beckmann P.
      The depolarization of electromagnetic waves.
      Golem, Boulder, Colorado, 1968.

\end{thebibliography}


\subsection{Jensen B: The Quantum Extension of the Drude-Zener Theory in Polar Semiconductors; p.169-188}
p.169-170:\\
\textbf{Introduction}\\
The classical Drude model for the complex dielectric constant of a semi-conductor can be used to
extrct the mobility and the free-carrier density $n_e$ from an analysis of the reflectivity and transmittance
data in the ar infrared (1-4)=
      \cite{Palik1979, Holm1977,Perkowitz1971,Perkowitz1974,Fan1967}.
The dielectric constant $\varepsilon$ is the square of the complex refractive
index, which determines the optical properties of a given material. One has
\begin{align}
   \varepsilon = \varepsilon_1 - i \varepsilon_2 = N^2,
   \label{compEps}
\end{align}
were the real and imaginary parts of the complex  parts of the complex dielectric constant $\varepsilon_1$ 
and $\varepsilon_2$ are functions of the complex refractive index N as
\begin{align}
   N = n -i k
\end{align}
\begin{align}
   \varepsilon = n^2 - k^2
\end{align}
\begin{align}
   \varepsilon = 2nk = \frac{4\pi \sigma}{\omega}
   \label{imEps}
\end{align}
The choice of $n-ik$ rather than $n+ik$ is determiend by the original use of 
$\exp i(\omega t - \boldsymbol q \cdot \boldsymbol r)$ in the assumed plane-wave solution of
Maxwell's equations.
In Eqs \eqref{compEps}-\eqref{imEps}, $n$ is the real part of the complex refractive index, $k$
the imaginary part or extinction coefficient, and $\sigma$ the optical conductivity. The
absorption coefficient $\alpha$ is proportional to $\sigma$, to $\varepsilon_2$, and to $k$:
\begin{align}
   n\alpha = \frac{4 \pi \sigma}{c} = \frac{\omega}{c} \varepsilon_2
\end{align}
\begin{align}
   n\alpha =  = \frac{\omega}{c}k = \frac{1}{\delta}
\end{align}
The extinction coefficient k is essentially the ratio of the free-space wavelength of light of 
frequency $\omega$ to the skin depth $\delta$.\\
%
The Drude theory gives the free-carrier contribution to $\varepsilon_1$ and $\varepsilon_2$ in terms
of the plasma frequency $\bar\omega_p$ and the electron scattering time $\tau$ as
\begin{align}
   \varepsilon_1 &= \varepsilon_{\infty} \frac{1 - \bar\omega_p^2}{\omega^2 \eta}\\
   \label{eps1Drude}
\end{align}
\begin{align}
   \varepsilon_2 &= \frac{\omega_p^2}{\omega^2 \eta} \frac{ 1 }{ \omega \tau}
\end{align}
where $\varepsilon_{\infty}$ is the high-frequencylattice dielectric constant The reail and imaginary
parts of the complex refractive index are obtained from $\varepsilon_1$ and $\varepsilon_2$.
One has
\begin{align}
   \varepsilon &= \sqrt{\varepsilon_1^2 + \varepsilon_2^2} = n^2 + k^2, \\
   n = \sqrt{\frac{\varepsilon + \varepsilon_1}{2} },
   k = \frac{\epsilon_2}{2n} = \sqrt{ \frac{\varepsilon - \varepsilon_1}{2}}.
\end{align}
Experimentally, $n$ and $k$ are found from measurements of the reflectivity R of a bulk, opaque sample
and the transmittance T of a slab, which are given in terms of $n$ and $k$ as
\begin{align}
   R &= \frac{(n-1)^2 + k^2}{(n+1)^2 + k^2} \\
   R &= \frac{ (1-R)^2 e^{-2\omega k d / c} }{ (1-R)^2 e^{-4\omega k d / c} },
\end{align}
where $d$ is the sample thickness. For the slab multiple-reflection effects are averaged,
so that interface fringes are not resolved.
\\
\\
p.171:\\ In the far infrared, for photon energues small compared with $k_0 T$ ($k_0$ boltzmans const.)
and with the energy $\hbar \omega_Q$ of the phonon involved in the scatteering, the
quantum result reduces to the $\lambda^2$ dependenve given by the Drude Theory, and the quasi classical
Boltzmann transport equation (1-3). The departurs from the Drude theory
at high frequencies are associated mainly with $k$ rather than $n$, and hence,
the transmission is affected more than the reflectivity. The latter depends on
$k$ in the region of the reflectivity minimum, where $n \simeq 1$, but is determined 
essentially by $n$ over the region of the absorption spectrum for which $n > k$,
which is the region where departures from the Drude theory would occur.
\\
(...)
\\
The response of electrons to a driving field may be followed from the wuasi-classical
limit of small $\omega$ to the quantum limit that occurs when $\hbar \omega$ is no longer small
compared with characteristic energies of the system. In this case, a generalized Boltzmann equation
is obtained that reduces to the quasi-classical Boltzmann transmport equation when the electron wave 
vector $q$ tends to zero and $\omega$ is small (14-17)=
\cite{Jensen1975,Price1966,Argyres1961,Kohn1958}
. When $\omega$ is appreciable, one obtaines,
under certain conditions, a solution of the Boltzmann equation in terms of a frequency-dependent
relaxation time. This relaxation rate, which has been tabulated as a function of frequency and 
carrier concentration for various materials (18-20)=
\cite{Jensen1977,Jensen1979,Jensen1981}
, can be used in the usual expression of the 
classical Drude theory to obtain the quantum result. In particular, the low-
frequency $\hbar\omega \simeq k_0T$ limit gives a good estimate for the dc mobility as a function
of carrier concentration. At high frequencies, in lightly doped materials in which
polar scattering dominates, $n\alpha$ is proportional to $\lambda^3$ and $\varepsilon_2$ and $k$
are proportional to $\lambda^4$ rather than $\lambda^3$. The real part of the dielectric constant
is givn approximately by the Drude-theory expression and $n \simeq \sqrt{\varepsilon_{\infty}}$
for $\bar\omega_p \ll \omega \ll G/\hbar$, where $G/\hbar$ is the frequency of the fundamental absorption
edge and $G$ is the direct-band-gap energy of the semiconductor. As $\omega$ approaches $G/\hbar$ there
is a small quantum-mechanical correction to $\varepsilon_1$ and hence to $n$.
A summary of the results of the quantum theory is given in Section II.
\\
\\
p.176:\\
\textbf{Comparison with experimental data}\\
A calculation of $\varepsilon_1$ appropriate to electrons in polar semiconductors with the
band structure of the Kane theory and based on the quantum density-matrix equation of motion yields a 
high frequency modification to Eq.\eqref{eps1Drude}. 
For $\omega \tau \gg 1$ and $X < 0.1$, ($X = \hbar \omega/G$) one obtains (18,22)
\begin{align}
   \varepsilon_1 &= \Bigg[ \frac{\varepsilon_{\infty}}{1 - X} \Bigg]\big[ 1 - (X/\varepsilon_{\infty}) - \bar\omega_p^2/\omega^2 \big] \\
                &= \Bigg[ \frac{\varepsilon_{\infty}}{1 - X} \Bigg]\big[ 1 - \bar\omega_p^2/\omega^2 \big] \\
                &= \varepsilon_{\infty}(1 - \bar\omega_p^2/\omega^2), \:\:\:\:\: X \ll 1
\end{align}
We note that $1/\varepsilon_{\infty} <\sim$ 0.1 and hence $X/\varepsilon \ll 1$, for compounds we 
consider, and this term can be neglected. In the limit $X \ll 1$, the quasi-classical high-frequency
Drude result is recovered, as required. For $X \sim 0.1$, there is a high-frequency correction given by
Eq.(some equation), which is used to calculate the numerical values of $\varepsilon_1 = n^2 - k^2$.
The major modification of the classical result is dispersion in $n$ as one approaches the fundamental
absorption edge \cite{Jensen1983}.


\begin{thebibliography}{9}
      \bibitem{Palik1979}
       Palik ED, Hold RT.
       Nondestructire evaluation of semiconductor materials and devices.
       Plenum 1979 (Chap. 7) 
      \bibitem{Holm1977}
         Holm RT, Gibson, Palik ED, 
         J. Appl.Phys. 48,212 (1977)
      \bibitem{Perkowitz1971}
         Perkowitz S. J.Phys.Chem.Solids 32, 2267 (1971)
      \bibitem{Perkowitz1974}
         Perkowitz S, Thorland RH, Phys.Rev. B9, 545 (1974)
      \bibitem{Fan1967}
         Fan HY.
         Semiconductors and semimetals.
         (R.K. Willardson and A.C. Beer eds.),
         Vol.3, Academic Press, New York 1067

      \bibitem{Jensen1975}
         Jensen B. Ann. Phys. 95,229 (1975).
      \bibitem{Price1966}
         Price PJ. IBM J. Res. Dev. 10,395(1966)
      \bibitem{Argyres1961}
         Argyres PN, J. Phys.Chem.Solids 19, 66 (1961)
      \bibitem{Kohn1958}
         Kohn W, Luttinger JM.
         Phys. Rev. 108, 590 (1957)
         Kohn W, Luttinger JM.
         Phys.Rev.109,1892(1958)

      \bibitem{Jensen1977}
         Jensen B. Phys. StatusSolidi. 86, 291 (1978);
         Jensen B. SolidState commun. 24, 853 (1977)
      \bibitem{Jensen1979}
         Jensen B. J. Appl Phys. 50, 5800 (1979)
      \bibitem{Jensen1981}
         Jensen B.
         Laser Induced damage in optical materials; 1980 (H.E. Bennet, A.J. Glass, A.H. Guenther,
         and B.D.Newman, eds.), p.416, National bureau of standards special
         publication 620, Boulder, Colorado, 1981.

      \bibitem{Jensen1983}
         Jensen B, IEEE J. 
         Quantum Electron. QE-18, 1361 (1982); 
         %
         Jensen B, Torabi A, IEEE J.Quantum Electron. QE-19, 448-457,877-882,1362-1365 (1983);
         %
         Jensen N, Torabi A, J.Appl.Phys. 54,2030-2035,3623-3625,5945-5949 (1983).
\end{thebibliography}





\subsection{Shashanka S. Mitra \\
Optical properties of nonmetallic solids for photon energies below the fundamental band gap}
Department of electrical engineering, University of Rhode Island.
\\
\\
p.263-267:\\
\textbf{Infrared Dispersion by plasmons}\\
In the preceeding discussion of absorption of infrared radiation by phonons in a solid, it was tacitly
assumed that the solid was an insulator. This assumption does not hold well for a narrow--gap semiconductor at
ordinary temperatures or for a doped semiconductor with a partially filled conduction or valence band.
The collective excitation of this free-carrier electron gas(plasma) in such a crystal will modify the
infrared absorption by phonons, as discussed earlier. The dispersion mechanism through which electromagnetic
radiation interacts with a solid should include the contribution of free-charge carriers in solids for which
their numbers are significant, in addition to contributions from bound electrons and phonons.
The dielectric response function of such a solid can now be written as 
\begin{align}
   \varepsilon = 1 + 4\pi(\chi_{BE} + \chi_L + \chi_{FC}),
\end{align}
where $\chi_{BE}, \chi_L$ and $\chi_{FC}$, respectively represent the bound electron,lattice,and
free-carrier contributions to the electrical susceptibility. For the spectral region of interest, we
are not concerned with teh bound-electron dispersion; thus this term, as usual, will be 
represented by a dispersion-free, high-frequency dielectric-constant term
\begin{align}
   \varepsilon_{\infty} = 1 + 4\pi \chi_{BE}.
\end{align}
An approximate expression for the dielectric response function for a free-electron gas in a solid can be
obtained from the classical Drude model in which a free electron of effective mass $m*$ and charge $e$
is displaced by an amound $\boldsymbol x$ as a result of interaction with the electric field 
$\boldsymbol E$, with the equation of motion
\begin{align}
   m* \ddot{\boldsymbol x} + m* \gamma \dot{\boldsymbol x} = e \boldsymbol E_0 e^{-i\omega t}
\end{align}
The damping term proportional to the velocity obviously represents the electron-phonon scattering in 
a phenomenological manner. Solving for $\boldsymbol x$, one obtains
\begin{align}
   \boldsymbol x = -e \boldsymbol E /m* \omega (\omega + i \gamma)
\end{align}
The polarization, defined as the electric-dipole moment per unit volume, is given by
\begin{align}
   \boldsymbol P = Ne \boldsymbol x,
\end{align}
where N is the carrier concentration. Recalling that 
\begin{align}
   \boldsymbol P =  \big[ (\varepsilon - 1)/ 4\pi \big] \boldsymbol E,
\end{align}
one readily obtains
\begin{align}
   \varepsilon_{FC} = \varepsilon_{\infty} - \frac{4 \pi N e^2}{m*\omega(\omega + i \gamma)}
\end{align}
for the dielectric response function due to a single-component plasma. In terms of the plasma frequency
defined as 
\begin{align}
   \omega_p = \sqrt{\frac{4\pi N e^2}{\varepsilon_{\infty} m*}}
\end{align}
$\varepsilon_{FC}$ becomes
\begin{align}
   \varepsilon_{FC} = \varepsilon_{\infty} \big[ 1 - \big(\omega_p^2/\omega(\omega + i \gamma)\big)\big].
\end{align}
For a number of semiconductors there exists a region of the infrared spectrum in which the free-carrier 
contributions dominates, and both the bound electron and lattice contributions to dielectric response
function are negligible. For a semiconductor, such a situation prevails in a region of the infrared
spectrum in which $\omega_g \ll \omega_p \ll \omega_{TO}$, where $\hbar \omega_g$ is the
electronic band gap. The analysis of optical response is also smpler in a region in which the reflection


\begin{thebibliography}{9}
      \bibitem{yy}
\end{thebibliography}
