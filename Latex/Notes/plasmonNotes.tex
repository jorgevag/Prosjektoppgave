\section{Plasmons}
Notes from Justin White 
http://large.stanford.edu/courses/2007/ap272/white1/

\textbf{Surface Plasmon Polaritons}
\begin{itemize}
\item
Surface Plasmon polaritons are collective longitudinal oscillations
of electrons near a material surface, strongly coupled to 
an electromagnetic wave.

\item 
Both bulk and surface plasmons have associated EM-waves, and can
consequently be described by Maxwell's equations. 

\item 
the coherent oscillations of electron motion can be encapsulated in
the dielectric constant of the material. 

\item
The basic form of the bulk and surface plasmon solutions are shown
below:
\begin{align}
E_{bulk} &= E_0 e^{k_x x - \omega t}\\
E_{spp} &= E_0e^{-\kappa |z|} e^{k_x x - \omega t}
\end{align}
The bulk plasmons are associated with purely transverse EM waves
($E\perp k$ and  $B \perp k$) and can only exist for 
$\omega < \omega_p$. 
For $\omega > \omega_p$: the wave-vector for bulk plasmons becomes
imaginary, giving an exponentially decaying wave, instead of a 
propagating wave. It is for this reason that most metals 
are highly reflective for visible light ($\omega <\omega_p$),
but transparent for ultraviolet light ($\omega > \omega_p$)

\item
Surface plasmons have an associated EM wave with both transverse 
and longitudinal field components. Such waves can only be excited
at the interface between a conductor and dielectric, and aretightly bound to the surface. The field reach their maximum at the interface 
($z=0$), and exponentially decay away from the surface.

\item 
The wave-vector of the surface plasmon mode $k_{spp}$ always lies
to the right of the free space wave-vector $k_0$ (in the dispersion 
diagram/relation), such that $\lambda_{spp} < \lambda_0$. this makes
 it impossible to directly launch a surface plasmon wave by
illumination with free-space radiation, because the free-space photons
simply do not have enough momentum to excite the surface plasmon.

As $\omega$ increases, $k_{spp}$ gets larger and larger, moving
further  away from $k_0$ (textit{??making it harder and harder 
for light to excite the surface plasmons??})d. As $k_{spp}$ increases,
the surface plasmon wave is more tightly bound to the surface.
This process has an upper limit of $\omega_{sp}$, the surface 
plasmon resonant frequency, which occurs when the dielectric 
constant of the metal and the dielectric have the same 
magnitude but opposite signs.

\item
\textbf{Excitation of Surface Plasmons}\\
High energy electrons that bombard a thin metalic film can
launch surface plasmons and a surface plasmons of a whole range
of wavelengths can be excited. However, only plasmons
far along the dispersion curve, where $k_{spp}$ is largest
are generally excited.
\\
\\
As mentioned previously, direct excitation of surface plasmons by 
free-space photons is not achievable because $k_{spp}$ is 
always greater than $k_0$; this can be seen from the dispersion
relation, where the surface plasmon dispersion relation always
lies to the right of the free space dispersion curve.\\
This can be overcome by back-side illumination through a material
with a higher index of refraction $n$, where the far field radiation
has a larger wave-vector ($k=nk_0$) 
(like done in a Kretschmann-Raether coupler) [5].
A surface plasmon will be efficiently excited when 
\begin{align}
k_{\parallel} = n k_0 \sin \theta = k_{spp}
\end{align}
A mor egeneral approach to launch surface plasmons with light
is the use of structured surfaces that are able to impart momentum
on the photon, enabling it to couple to the surface plasmon mode.
Anything from a single sub-wavelength disk or slit, to rectangular
or sinusoidal diffraction gratings are used for this type of coupling.
\textbf{A thorough overview of surface plasmon coupling 
and patterned and rough surfaces is given by Raether[6]}


\item Appendix: Derivation of Bulk and Surface Plasmons (see article).
\end{itemize}

Other Nice sources:
\begin{itemize}
\item Really nice article on Plasmons:\\
http://nanocomposix.com/pages/plasmonics

\item Article:\\
http://cdn.intechopen.com/pdfs-wm/44351.pdf

\item Surface Plasmon Theory(book?):\\
https://www.physik.hu-berlin.de/de/nano/lehre/Gastvorlesung%20Wien/Plasmonics%20Pitarke

\item Chemistry-blog:\\
http://www.chemistry-blog.com/2007/03/19/plasmonics-part-ii/

\item Mie theory: \\
http://www.orc.soton.ac.uk/publications/theses/1460T\_lnn/1460T\_lnn\_03.pdf
\end{itemize}




\newpage
\textbf{From Wikipedia}
\begin{itemize}
   \item \textbf{What are Plasmons?} \\
A plasmon is a quantum of plasma oscillation (quasiparticle from the quantization of plasma oscillations).
Plasmons are collective (a discrete number) oscillations of the free elctron gas density.
Plasmons can also couple with a photon to create another quasiparticle called a plasma polariton
(electromagnetic wave - electric/magnetic dipole-carrying exitation - coupling.
\\
\\
Plasmons can be described as oscillations of free elctron density with respect to
fixed positive ions in a metal. Imagine a cube of metal placed in an external 
electric field pointing to the right. Electrons will move the the left side and uncover
positive ions on the right side. The electrons will continue moving left until they
cancel the field inside the metal. Removing the field will make the electrons move back by their
mutual repulsion and attraction to the ions, leaving the electrons to oscillate back and forth,
at the \textbf{plasma frequency}, in a so called plasma oscillation.

\item \textbf{Plasma Oscillation, aka ''Langmuir waves''} \\
Rapid oscillations of electron density in conducting media such as plasmas or metals.
The oscillations can be described as an instability in the dielectric function of a free electron gas.
The frequency depends weakly on the wavelength of the oscillation.
\\
\\
'Cold' electrons (plasma oscillations)\\
If the thermal motion of the electrons is ignored and assuming infinite ion mass,
the charge density oscillates at the plasma frequency
\begin{align}
   \omega_{pe} &= \sqrt{\frac{n_e e^2}{m^* \varepsilon_0}}, \text{[rad/s] (SI-units)} \\
   \omega_{pe} &= \sqrt{\frac{4 \pi n_e e^2}{m^*}}, \text{(cgs-units)},
\end{align}
where $n_e$ is the number density of electrons, $e$ is the electric charge, $m^*$ is the effective mass of
the electron and $\varepsilon_0$ is the permittivity of free space. Since the frequency is independent of 
the wavelength, these oscillations have an infinite phase velocity and zero group velocity.
Note in addition that, if $m^*$ is the electron mass $m_e$, the plasma frequency $\omega_{pe}$
depends only on the physical constants and concentration of electrons $n_e$. The numeric expression is:
\begin{align}
f_{pe} = \frac{\omega_{pe}}{2 \pi} \approx 8980 \sqrt{n_e}, \text{[Hz]}
\end{align}
with $n_e$ in [cm$^{-3}$]
\\
\\
'Warm' electrons (plasma oscillations)\\
When the effects of the electron thermal speed $v_{e,th} = \sqrt{\frac{k_B T_e}{m_e}}$ are taken into account,
the electron pressure acts as an additional restoring force and the oscillations propagate with
frequency and wavenumber related by the longitudinal Langmuir wave:
\begin{align}
\omega ^2 = \omega_{pe} ^2 + \frac{3 k_B T_e}{m_e} k^2 = \omega_{pe} ^2 + 3k^2 v_{e,th}^2
\end{align}
called the 'Bohm-Gross dispersion relation'. If the spatial scale is large compared to
the Debye length (measure of a charge carrier's net electrostatic effect in solution, 
and how far those electrostatic effects persist), the oscillations are only weakly modified
by the pressure term, but at small scales the pressure term dominates and 
the waves become dispersionless with a speed of $\sqrt{3}v_{e,th}$. For such waves, however, the
electron thermal speed is comparable to the phase velocity, i.e.
\begin{align}
v \sim v_{p,th} \equiv \frac{\omega}{k},
\end{align}
so the plasma waves can accelerate electrons that are moving with speed nearly equal to the
phase velocity of the wave. This process often leads to a form of collisionless damping
called Landau damping. Consequently, the large-k portion in the dispersion relation is difficult to
observe and seldom of consequence.
\\
\\
In metal of semiconductor, the effect of the ions periodic potential must be taken into account. This is
usually done by using the electrons effective mass in place of $m$.

\item \textbf{Role of Plasmons} \\
Plasmons play a large role in the optical properties of metals. Light of frequencies below the
plasma frequency is reflected, because the electrons in the metal screen the electric field 
of the light. Light of frequencies above the plasma frequency is transmitted, because the
electrons cannot respond fast eneough to screen it. In mot metals, the plasma frequency is in the 
unltraviolet, making them shiny (reflective) in the visible range. 
In semiconductors, the valence electron plasma frquency is usually in the deep ultraviolet, which is why
they are reflective.
\\
The plasmon energy can often be estimated in the free electron model as
\begin{align}
   E = \hbar \sqrt{\frac{n_e e^2}{m^* \varepsilon_0}} = \hbar \omega_p
\end{align}




\item \textbf{Surface Plasmons (SPs)} \\
Surface plasmons (plasmons at the interface of two materials) interact strongly with light,
resulting in a polariton (usually occurs at metal or doped dielectric interface, 
which both have small Im($\varepsilon$) > 0 and big Re($\varepsilon$) < 0).
These surface electron oscillations can exist at the interface between
any two materials where the real part of the dielectric function changes sign across the interface
(e.g. a metal-dielectric interface like metal sheet in air).
\\
SPs have lower energy than \textbf{bulk (or volume)} plasmons, which quantise the longitudinal
electron oscillations about positive ion cores within the bulk of an electron gas (or plasma).
\\
\\
The charge motion in a surface plasmon always create electromagnetic fields outside (as well as inside)
the metal. The total excitation, including both the charge motion and associated electromagnetic field,
is called either a \textbf{surface plasmon polariton} at a planar interface, or a \textbf{localized
surface plasmon} for the closed surface of a small particle.
\\
\\
Surface Plasmon polaritons can be excited by electrons or photons. In the case of photons, it
cannot be done directly, but requires a prism, or a grating, or a defect on the metal surface.
\textit{??? Or like trunctated spheres on granular films???}.
\\
\\
At low frequency an SPP approaches the dispersionrelation in free space $\omega = ck$.
At high frequency, the dispersion relation reaches an asymptotic limit called the ''sufrace plasma frequency''.
\\
\\
As an SPP propagates along the surface, it loses energy to the metal
due to absorption and due to scattering into free-space or into other directions. The electric field
falls off evanescently perpendicular to the metal surface. At low frequencies, the SPP penetration depth into
the metal is commonly approximated using the 'skin depth formula'. In the dielectric, the field 
will fall off far more slowly. SPPs are very sensitive to slight perturbations within the skin depth and because of this, SPPs, are often sed to probe inhomogeneities of a surface
\\
\\
Surface plasmons have been used to control colors of materials and is possible since controlling 
the particle's shape and size determines the types of surface plasmons that can couple to it and 
propagate across it. This in turn controls the interaction of light with the surface.
These effects are illustrated by the historic \textit{stained glass} wich adorn medieval
cathedrals. In this case, the color is given by metal nanoparticles of a fixed size which interact
with the optical field to give the glass its vibrant color. To produce optical range surface plasmons
effects involves producing surfaces wich have features < 400nm.
\\
\\
Surface plasmons are very sensitive to the properties of the materials on which they propagate.


\item \textbf{Surface Plasmons Resonance (SPR)} \\
Surface plasmon resonance is the resonant oscillation of conduction electrons at the interface
between a negative and positive permittivity meterial stimulated by incident light. The resonance
condition is estalished when the frequency of incident photons matches the natural
frequency of surface electrons oscillating against the restoring force of positive nuclei.
\\
\\
Surface plasmon polaritons are surface electromegnetic waves that propagate in a direction parallel to
the metal/dielectric (or metal/valcuum) interface. Since the wave is on the boundary of the metal
and the external medium, these oscillations are very sensitive to any change of this boundary, such as
adsoption of molecules to the metal surface.
\\
\\
To describe the existence and properties of surface plasmon polaritons, one can choose from various models,
e.g. the \textbf{Drude Model}. The simplest way to approach the problem is to treat each
material as a homogeneous continuum, described by a frequency-dependent relative permittivity between 
the external medium and the surface (this is a complex dielectric function). In order
for the terms that describe the electronic surface plasmons to exist, the real part of the dielectric constant
of the metal must be negative and its magnitude must be greater than that of the dielectric.
This condition is met in the infrared-visible wavelength region for air/metal and water/metal interfaces (where
the real dielectric constant of a metal is negative and that of air or water is possitive).
\\
\\
Localized SPRs (LSPRs) are collective charge oscillations in metallic nanoparticles that
are excited by light. They exhibit enhanced near-field amplitude at the resonance wavelength.
This field is highly localiced at the nanoparticle and decays rapidly away from the nanoparticle/dielectric
interface into the dielectric background, though far-field scattering by the particle is also 
enhanced by the resonance. Light intensity enhancement is a very important aspect of LSPRs and
and localization means the LSPR has very high spatial resolution (subwavelength), lmited
only by the size of nanoparticles. Because of the enhanced field amplitude, effects that depend on the
amplitude such as magneto-optical effect are also enhanced by LSPRs.
\\
\\
In order to excite surface plasmoms in a resonant manner, one can use an electron or light beam
(visible and infrared are typical). The incoming beam has to match its momentum to that of the plasmon.\\
With p-polarization this is possible by assing the light through a block of glass to
increase the wavenumber (and the momentum) and achieve the resonance at a given wavelength and angle.\\
s-polarized light however cannot excite electronic surface plasmons.
\\
\\
When the surface plasmon wave interacts with a local particle or irregularity, such as a rough surface,
part of the energy can be re-emmited as light. This emitted light can be detected behind the metal film
from various diretions.


\item \textbf{The Drude Model}\\
Treats the behavior of electrons in a solid like a pinball machine. 
The electrons are small light balls in a sea of static, positively charged ions. The only 
form of action instantaneous collisions.


\item \textbf{Mie Scattering}\\
Mie theory is sometimes used for the collection of methods and solutions to Maxwell's equations
for scattering, by e.g. using geometries where one can write separate equations for the radial and angular
dependence of solutions. More broadly, ''Mie scattering'' suggests situations where the size of the
scattering particles is comparable to the wavelength of the light, rather than much smaller or much larger.
\end{itemize}

\begin{thebibliography}{9}

      \bibitem{} 

\end{thebibliography}

