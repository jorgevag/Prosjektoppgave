\section{\textbf{Optical Properties of Surfaces Notes}}
\begin{itemize}
\item $\rho$ is the number of island per unit of surface area.
\item $\alpha_{\parallel}^{10}$: (averge) parallel quadrupole polarizability \\
   $\alpha_{\perp}^{10}$: (average) normal quadrupole polarizability
   \item 
   \begin{align}
      \tau = -\rho \alpha_{\parallel}^{10} \:\:\:\text{ and }\:\:\: \delta = -\rho [\alpha_{\perp}^{10} + \alpha_{\parallel}^{10} ]/\varepsilon_a
   \end{align}
   $\varepsilon_a$ is the ambient/surrounding dielectric constant.

   \item an extremely important difference between an interface and the bulk of the adjacent phases is the
   inherent asymmetry of the interface in its responce to fields normal to the surface or parallel
   to the interface. This is eminently clear for a thin metal film between two dielectric media.
   If one applies an electric field along the film this results in a current while 
   an electric field normal to the film does not lead to a current. (p.7)

   \item excess quantities which not affect the boundary conditions have no relevance for the reflection
   and transmission of light by the interface.

   \item The resistance of the sublayers are put in series and add up for an electric field perpendicular
   to the surface. This is very different from the case described above for an electric field parallel
   to the layer where the conductivities of the sublayers were put in parallel. 
   \\
   \\
   If the electric field is parallel to teh layer, one finds an important contribution from the layer if its 
   conductivity is large compared to the surrounding medium. \\
   For a field orthogonal to the layer, however, one finds an important contribution if the conductivity
   of the layer is much smaller than the conductivity of the surrounding medium. It is this very characteristic
   difference in the response of the layer to fields orthogonal and parallel to teh layer which is the
   origin of many of the interesting electromagnetic properties of the layer. (p.10)

   \item If the electric field is parallel to the layer the dielectric constants of the sub-layers are so
   to say put in parallel. If the elextric field is normal to the layer, however, they are put in series.
   Furthermore one finds that, if the dielectric constant of the layer is much larger than the dielectric
   constants of the surrounding homogeneous media, the constitutive coefficient $\gamma_e$ is much smaller 
   than the coefficient $\beta_e$ and dominates the behaviour of the layer. IF the dielectric constant
   of the layer is much smaller than the dielectric constants of the surrounding homogeneous media,
   the constitutive coefficient $\beta_e$ is much larger than the coefficient $\gamma_e$ and dominates the
   behaviour of the layer. It is again this very characteristic difference in the response of the layer
   in the directions orthogonal and parallel to the layer which is the origin of many of the
   interesting electromagnetic properties of the layer. (p.13)

\item the jump in the extrapolated fields are given in terms of the total excess of 
$D_{ex,\parallel}$,$E_{ex,z}$,$B_{ex,\parallel}$,$H_{ex,z}$,$I_{ex,\parallel}$ and $\rho_{ex}$.
Notice that it is however not affected by the total excesses of
$D_{ex,z}$,$E_{ex,\parallel}$,$B_{ex,z}$,$H_{ex,\parallel}$,$I_{ex,z}$ and $\rho_{ex}$;
these total excesses tehrefore have no effect on the reflection and transmission
amplitudes and may as such be neglected in the descripttion of the optical properties of the 
interface.(?I think this is for $k_z = 0$?). (p.16)

\item p.21 - 72. NOT READ

\item If one consider a single island surrounded by the ambient the electro-magetic response may be
characterized by dipole, quadrupole and higher order multipole polarizabilities. If this particle
is brought close to the substrate the multipole polarizabilities are modified, due to an 
induced charge distribution on the surface of the substrate, and become dependent on the distance to
the substate. (p.73)
\item neighboring island interaction also changes the polarizabilities of the island and result in 
polarizabilities parallel and othogonal to hte surface which are unequal, even if the islands are spherical.
As a consequence the film is anisotropic in its reaction to fields parallel versus fields normal to the 
surface. Further contribution to this anisotropy is achieved if the particle is not symmetric.
(p.73)

\item The resulting dipole polarizabilities of the islands, parallel and orthogonal to the
surface , are directly related to teh constitutive coefficients $\gamma_e$ and $\beta_e$, respectively.
The susceptibilities $\delta_e$ and $\tau$ account for the fact that the dipoles are situated at a
finite distance from the surface of the substrate. It follows in the context of the polarizable
dipole model that $\beta_e$ and $\gamma_e$ are proportional to the typical diameter of the islands times the
coverage. The coverage is the surface area of the particles as viewed from above divided by
the total surface area. For identical spheres the coverage is equal to $\pi r^2 \cdot N_{islands}/A$.
Furthermore it follows that $\delta_e$ and $\tau$ are proportional to the diameter times the distance
of the center of the islands to the substrate times the coverage.\\
In general, and not only in the polarizable dipole model, one may in fact use the ratio of the coefficients
$\delta_e$ and $\tau$ with the coefficients $\gamma_e$ and $\beta_e$ as a rough measure for the distance
of the center of the islands to the substrate. (p.74) 

\item  the first term ($A_{lm}$ term) is due to sources located in the region r < a.
The second term ($B_{lm}$ term) is due to sources in the region r > b and may be identified
as the incident field due to external sources. The first term is due to the charge distribution induced
in the island. The island has no net charge $\rightarrow$ the l = 0 contribution in the first term
is zero. $A_{lm}$ gives a contribution to the potential which decaus as $r^{-l-1}$, and may be
identified (apart from a numerical constant) as an amplitude of the $l$'th order mutlipole field. 
\\
The amplitude of these multipole fields are given in terms of the amplitudes of the 
incident field using a polarizability matrix:
\begin{align}
A_{lm} = - \sum\limits^{'}_{l',m'} \alpha_{lm,l'm'} B_{l'm'} \:\:\:\:\text{ for } l \neq 0
\end{align}
where the prime over sum indicates that $l' \neq 0$.
   \begin{itemize}
      \item polarizable dipole model:  all amplitudes with $l$ and $l' \neq 1$ is set to zero.
      \item polarizable quadrupole model: all amplitudes with $l$ or $l' > 2$ is set to zero.
   \end{itemize}
   (p.79)

\item s. 80 viser en del intuitiv forklaring av hvordan polariseringen påvirker feltet i øvre og nedre
medium. Kan hjepe å forstå hvordan de første $A_{lm}$ amplitudene representerer dipolbidraget.
\item For particles with symmetry axis normal to the surface of the substrate ($A$ characterizses the strength
of the reflected dipole):
\begin{align}
&\alpha_{\parallel}(0) = [1 + A \alpha_{\parallel}]^{-1} \alpha_{\parallel} \\
&\alpha_{\perp}(0) = [1 + 2 A \alpha_{\perp}]^{-1} \alpha_{\perp}
\end{align}
If there are two two different polarizabilities parallel to teh surface the above equation is valid for 
both of them. Furthermore, regarding the anisotropy in the interaction with the substrate, even
if the particle is spherical, so that $\alpha_{\parallel} = \alpha_{\perp}$, the different response
to fields along the surface and normal to the surface results in $\alpha_{\parallel}(0) \neq \alpha_{\perp}(0)$.
(p.81).
\\
\\
If one covers the substrate with a low density of identical islands, which have a rotational symmetry
around the normal of the surface, one finds for the first order interfacial susceptibilities
\begin{align}
\gamma_e(d) = \rho \alpha_{\parallel}(0) = \rho[1 + A \alpha_{\parallel}]^{-1} \alpha_{\parallel}\\
\beta(d)_e = \rho \varepsilon_a^{-2}\alpha_{\perp}(0) = \rho \varepsilon_a^{-2}[1 + 2 A \alpha_{\perp}]^{-1} \alpha_{\perp},
\end{align}
where $\rho$ is the number of particles per unit of surface area. The argument $d$ indicates
that the location of the dividing surface is chosen to be the $z=d$ surface, which in this case
coincides with the surface of the substrate. One important thing: in writing these formulae the polarization
due to the dipoles is in fact taken to be located at the surface of the substrate. The origin of $tau$ and
$\delta_e$ is related to a proper choice of this location.

\item The reason, that it is also in the dipole model important to take 
finite values for $\delta_e$ and $\tau$, is related to the occurence of phase factors. Light reflected from
the film can either be reflected directly from the island film or be transmitted and subsequently
reflected by the surface of the substrate. It is clear that the phase difference between these
two contributions can be important. The analysis on p. 83 and the resulting finite value of 
$\delta_e$ and $\tau$ account for the phase differences between the actual location of the islands
and the surface of the substrate to linear order. (p.83)
\item $\delta_e$ and $\tau$ have relatively small modification of the optical properties, but contain 
nevertheless interesting new information. The coefficients $\gamma_e$ and $\beta_e$, which usually
have a much larger influence, are a measure of the amount of material deposited on the surface 
of the substrate. They are therefore a measure of what on usually calls teh weight thickness.

\item (p.95)\\
   Incident e field $ \boldsymbol E_0$ and the corresponding potential 
$\psi_0(\boldsymbol r) = - \boldsymbol E_0 \cdot \boldsymbol r$ is given on .

\item Appendix A (p. 106): The derivation of the surface constitutive coefficients $\gamma_e(d)$ and 
   $\beta_e(d)$. for an identical particle, island film, in the low density limit 
   \\
   \\
   Instead of dipoles, small dielectric spheres are considered, 
   located a distance $d \gg R$ above the substrate (substrate is located at $z = d$).
   Excess fields are introduced and the total field in the ambient is the sum of the external field,
   the dipole fields of the spheres (with centers in $(\boldsymbol R_{i,\parallel},0)$) and the
   image dipoles (with centers in $(\boldsymbol R_{i,\parallel},2d)$).

\item (p.117)\\
   When the islands are broughth close to the substrate, one must also account for the 
   modification of their dipole and quadrupole polarizabilities due to the interaction with higher order 
   multipoles in the substrate. It is sufficient to calculate only the modification of these 
   polarizabilities, as long as one is interested in the constitutive coefficients $\gamma_e$, $\beta_e$,
   $\delta_e$ and $\tau$. 
\item (p.118)\\
   The dipole and the quadrupole approximation is in general inadequate for islands which are flat, i.e.
   islands for which teh distance of the center to the surface of the substrate is smaller than 30\% of
   the linear diameter along teh surface. 
   
\item p.173

\item p.204\\
   (I think this is for the dipole + quadrupole expansion.. maybe) \\
   The calculations were done for scaled densities $\hat \rho \equiv \rho(2R)^2 = (2R/L)^2 = (2\hat R)^2$ of
   0 (no interaction), 0.2, 0.4, and 0.8. L is the lattice cosntant and $\hat R \equiv R/L$. The reason
   to go to a higher order in the multipole expansion in the interaction with the substrate is that,
   close to touching the substrate, the higher order multipoles start to give corrections of about 5 to 10\%.
   As is analyzed in great detail by Haarmans and Bedeaux (49),\textbf{higher order multipole corrections 
   for the interaction along the substrate become important for $\hat \rho > 0.9$. They use as density
   parameter the coverrage, which is equal to $(\pi/4)\hat \rho$ for the square array.}
   (...) The interaction along the substrate has a substantial effect on the size of the polarizabilities and 
   the resulting constitutive coefficients. The polarizability along the substrate approximately doubles 
   from $\hat \rho = 0$ to 0.8 (?for triangular lattice?), while the polarizability normal to the surface
   decreases by a factor of 2. The resonance frequency shifts a little bit down for the polarizability
   parallel and up for the polariability normal to the surface. The contributions due to the 
   quadrupole polarizabilities to the coefficients $\hat \delta_e$ and $\hat \tau$ are small compared to
   the contributions due to the dipole polarizabilities. \\
   \\
   p.209\\
   These coefficient are therefore mainly due to the shift of the induced dipole to the surface of 
   the substrate.
\item p.209-210.\\
   \textbf{Here (also something about square lattice), they talk about that the distance between the
   islands must be so far apart to give reasonable results.}It also seems like they're saying that
the polarizabilities and constitutive coefficients of OBLATE SPHEROIDS do not depend very much on the
coverage. The reason is that the center of the OBLATE SPHEROIDS remain relatively far apart, even for high
densitites. \\
For PROLATE SPHEROIDS the dependence on the coverage becomes dramatic, reason being that they approach each
other for higher coverage to distances smaller than the long axis. It is to be expected that these results
are unreliable for the combination of a high density and a small axial ratio\\
\\
If one would set a criterium that the distance between the particles should be larger than half the 
long axis, one finds that for densities 0.0,0.2,0.4 and 0.8 the maximum values of 1/(1+AR) are
1.00,0.69,0.61 and 0.53 respectively. It follows that all the new maxima appearing for larger values of 
$\hat rho$ and 1/(1+AR) are outside the range of reliability of the dipole approximation for the 
interaction along the substrate. Since the shape of the particles in island films is usually oblate
rather than prolate, the dipole approximation to describe the interaction between the particles 
will in practice suffice.

\end{itemize}
