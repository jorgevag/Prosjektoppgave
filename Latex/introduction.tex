\section{Introduction}
Today, in the light of global warming, the importance of reducing the amount released
greenhouse gases into the atmosphere is more crucial now than ever. But because the use of
fossil fuel is so well implemented in our society and constitute such an important energy source
it is difficult to find replacements. However, through identifying inefficient systems
that constitute the worst energy consumers and making use of technologies to increase their efficieny,
a significant amount of the energy related carbon dioxide emissions may be reduced.
%In the light of todays problems regarding global warming, 
%techlonogies that lower the overall energy consumptions, and thereby decrease the resulting
%energy-related carbon dioxide emmision, are more important now than ever. 
%

%A significant amount of the total energy consumption in developed contries are due to buildings,
%consuming 30-40\% of the contry's total energy usage. 
A relatively large portion of the energy consumption worldwide is in fact due to
the building sector. 
Today the energy consumption of buildings in developed contries constitute about 30-40\%
of their total energy usage. In humid regions, this increases to roughly 30\% to 50\% 
\cite{AlRabghi2001, Wilde2004, Kwak2010}. 
In 2010, 41\% of the primary energy
of the U.S. (being the second largest energy consumer globally), accounting for 7\% of the global energy use,
were consumed by the building sector.
This resulted in approximately 40\% of the total energy-related carbon dioxide emmision in the US. 
For comparison, the building sector in China accounted for 18\% of the CO$2$ emmision of the country, 
whereas worldwide the building energy consumption is the cause of 8\% of the total carbon dioxide emmision.
\cite{buildingsEnergyDatabook, Hong2009}.
Because this in total makes such a significant impact worldwide, it should motivate 
measures to be taken to reduce the building energy consumption, in order to reduce the 
related CO$_2$ emmisions. In addition to lowering the energy consumption, it would also lower the 
energy related costs related to the use of electricity. So reducing the 
overall building energy usage would have both economical and environmental benefits. 
\\
\\
%Considering the energy consumption of buildings, devices which stand for
%heating, ventilation and airconditioning (HVAC) help to maintain a comfortable indoor climate. 
%HVAC devices are compensating for heat loss through the building's envelope 
%(walls, roof, windows or any element separating
%the indoor from the outdoor) or excessive heating due to the thermal radiation from the sun.
%Together with lighting, HVAC were responsible for about 60\% of the total building energy consumption 
%in 2010
When considering the energy consumption of buildings, devices which stand for
heating, ventilation and airconditioning (HVAC), which help to maintain a comfortable indoor climate,
are actually accounting for much of the overall energy usage. 
HVAC devices are used to compensate for heat loss through the building's envelope 
(walls, roof, windows or any element separating the indoor from the outdoor) 
or excessive heating due to the thermal radiation from the sun.
Together with lighting, HVAC were responsible for about 60\% of the total building energy consumption 
in 2010
\textbf{(???in the U.S. or in general???)}.
\cite{buildingsEnergyDatabook}. 
%
%Considering thermal loss through the building envelope, 
%the window elements are in fact the most energy 
%inefficient components \cite{Baetens2010} and improving their thermal properties will be crucial in order
%to reduce the electricity costs.
The approaches to increase the energy efficiency of buildings can be divided into two 
categories: active strategies, 
including improving HVAC systems and lighting of the building; and passive strategies, like
improving the thermal properties of the building envelope. The latter includes
measures like thermal insulation to walls, cool coatings on the roof tops
%Does this mean: reflective coatings on the roof for cooling? 
or using windows with special coatings, altering the optical properties of the window 
\cite{Bojic2001,Cheung2005,Synnefa2007,Sadineni2011}.
%\cite{Bojic2001} \cite{Cheung2005} \cite{Synnefa2007} \cite{Sadineni2011}.
%(\cite{Kamalisarvestani2013},7-10). 
%
%In fact, the windows are the most energy inefficient component
%of the building \cite{Baetens2010}. 
%Therefore, to effectively reduce electricity costs and the resulting CO$_2$ emmision,
%one should focus on improving the thermal performance of the window.
%%Therefore, improving the thermal performance of the window 
%%will result in reduced electricity costs and related emission of greenhouse gases
With the window actually being the most energy inefficient component of the building 
\cite{Baetens2010} 
%
\textbf{<- burde sjekkes, men har ikke tilgang (har ikke prøvd VPN enda}
%
, the measures focusing on improving the thermal properties 
of the window seems most reasonable, as they try to solve the seemingly 
worst issue regarding the energy efficiency of the building envelope.
\cite{Kamalisarvestani2013}.
\\
\\
The thermal properties of the window depend mainly on
the outdoor conditions, like shading, building orientation and type, in addition to the
area of the window, its glass properties and glazing characteristics \cite{Hassouneh2010}. In window standards,
the latter is the most important, because the glazing characteristics includes thermal transmittance
coefficient \cite{Tarantini2011}.
(\cite{Kamalisarvestani2013}, s.354 avsnitt 2)\\
One way of improving the thermal efficieny of the window is to add some additional mechanism,
allowing the window to change its properties to the environment. An example of such improved 
windows are called ''smart-'' or ''intelligent windows'' and will be discussed in the next section.
\\
\\
(The information for this section was gathered by \cite{Kamalisarvestani2013}) \textbf{FJERN DETTE eller 
INKLUDER DETTE PÅ EN BEDRE MÅTE (om det er verdt å nevne) }

\cite{buildingsEnergyDatabook}
\begin{thebibliography}{9}


      %Main article, for now..., for now...
      \bibitem{Kamalisarvestani2013}
      Kamalisarvestani M, Saidur R, Mekhilef S, Javadi FS.
      \emph{Performance, materials and coathing technologies of thermochromic thin films on smart windows}, 
      PressOrSomething?, 
      Renewable and Sustainable Energy Reviews ??
      2013; 26:353-364 ??
      \textbf{ER DETTE RIGKTIG?}

      \bibitem{buildingsEnergyDatabook}
      \emph{DoE U. Buildings energy databook}
      Energy Effucuebct \& Renewable Energy Department 2011.
      \textbf{MÅ SJEKKES!}

      \bibitem{AlRabghi2001}
      Al-Rabghi OM, Hittle DC.  
      \emph{Energy simulation in buildings: overview and BLAST example.} 
      Energy Conversion and Management 
      2001;42(13):1623-35 
      \textbf{MÅ SJEKKES!}

      \bibitem{Wilde2004}
      Wilde PD, Voorden MVD.C.  
      \emph{Providing computational support for the selection of energy saving building components.} 
      Energy and Buildings 
      2004;36(8):749-58

      \bibitem{Kwak2010}
      Kwak SY, Yoo SH, Kwak SJ.
      \emph{Valuing energy-saving measures in residential buildings: a choice experiment study.}
      Energy Policy
      2010; 38(1):673-7

      \bibitem{Hong2009}
      Hong T. 
      \emph{A close look at the China design standard for energy efficiency of public buildings.}
      Energy and Buildings
      2009;41(4):426-35

      \bibitem{Baetens2010}
      Baetens R, Jelle BP, Gustavsen A.
      \emph{Properties, requirements and possibilities of smart windows for dynamic daylight and solar energy control in buldings: a state-of-the-art review}.
      Solar energy Materials and Solar Cells
      2010;94(2):87-105

      \bibitem{Hassouneh2010}
      Hassouneh K, Alshboul A, Al-Salaymeh A.
      Influence of windows on the energy balance of apartment buildings in Amman.
      Energy Conversion and Management 2010;51(8):1583-91.

      \bibitem{Tarantini2011}.
      Tarantini M, Loprieno AD, Porta PL.
      A life cycle approach to Green Public Procurement of building materials and elements: 
      a case study on windows.
      Energy 2011;36(5):2473-82.

      %(ref 7-10 kamali)
      \bibitem{Bojic2001}
      Bojic M, Yik F, Sat P. 
      Influence of thermal insulation position in building envelope on the space cooling of high-rise 
      residential buildings in Hong Kong.
      Energy and Buildings 2001;33(6):569-81
      
      \bibitem{Cheung2005}
      Cheung CK, Fuller R, Luther M.
      Energy-efficient envelope design for high rise apartments.
      Energy and Buildings 2005;37(1):37-48

      \bibitem{Synnefa2007}
      Synnefa A, Santamouris M, Akbari H. 
      Estimating the effect of using cool coating on energy loads and thermal comfort in residential 
      buildings in various climatic conditions. 
      Energy and Buildings 2007;39(11):1167-74

      \bibitem{Sadineni2011}
      Sadineni SB, Madela S, Boehm RF. 
      Passive building energy savings: a review of building envelope components.
      Renewable and Sustainable Energy Reviews 2011;15(8):3617-31
\end{thebibliography}
