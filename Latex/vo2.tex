\subsection{Vanadium dioxide VO$_2$; A promising candidate} \label{sec:vo2}
The most promising thermochromic material regarding thermochromic intelligent windows
is vanadium (IV) oxide.
%
%This collowing block is from blackman (along with the currents referances)
Vanadium oxide can exist in four polymorphic forms; monoclinic, rutile and two 
metastable forms (2). The metal to semiconductor phase transition from the 
monoclinic to the rutile state occurs at 68$^{\circ}$C and is fully reversble (3). 
In this transition, large changes in electrical conductivity and optical properties in the near-IR region 
occur (4), \textbf{while the change in optical region is relatively small \cite[p.~395]{Parkin2006}} . 
Above $T_t$ it behaves as a semi-metal, reflecting to a wide range of solar wavelengths
, while below, it behaves as a semiconductor, reflecting significantly less in the near infrared region 
\cite[p.~4565]{Blackman2009}.%(omskrevet av meg).
\\
\\
The transition temperature of 68$^{\circ}$C is relatively low compared other thermochromic materials.
%The most promising thermochromic material regarding thermochromic intelligent windows
%is vanadium (IV) oxide, with a relatively low transition temperature of 68$^{\circ}$C.
68$^{\circ}$C is, however, far from the comfortable temperature region around $\sim 20^{\circ}$C.
This would leave the window in its monoclinic state for all natural ambient temperatures, 
never switching it's state and leaving it unsuited as a smart coating
\cite[p.~358]{Kamalisarvestani2013} \cite[p.~39]{Kanu2010}.
%"The most promising thermochromic material for window glazing is vanadium (IV) oxide, 
% owing to its relativity low
%transition temperature of 68◦C; however, this is still not low enough for it to
%be actually used on a larger scale. The introduction of some particular dopents
%lowers the Tc to a suitable temperature". 
%\cite{Blackman2009} %(direkte sitat)
%Vanadium oxide can exist in four polymorphic forms; monoclinic VO2(M) and rutile VO2 (R) and two 
%metastable forms VO2(A) and VO2(B) (2). The monoclinic state converts to the rutile state at 68 °C 
%through a fully reversble metal to semiconductor phase transition (MST) (3). 
%In this transition, large changes in electrical conductivity and optical properties in the near-IR region 
%occur (4), and behaves as a semi-metal, reflecting to a wide range of solar wavelengths
%above $T_t$, while below it behaves as a semiconductor, reflecting 
%significantly less in the near infrared region. \cite[p.~4565]{Blackman2009} %(omskrevet av meg)
%
\\
\\
\cite{Kamalisarvestani2013}:\\
\textbf{TALK ABOUT DOPING}
(kamalisarvestani: The size and charge (84,86,87) of dopant ion, film's strain (88,89) as well as
variations in electron carrier density are the determinant factors prevailing on the fall or rise of
$T_t$ (90))
%
Many studies regarding vanadium dioxide coatings have reported low transmittance in the visible range
as discussed in the review of Kamalisarvestani et al. \cite[p.358]{Kamalisarvestani2013} and 
constitutes a the most critical weakness of VO$_2$ coatings. The switching efficiency $\eta_T$,
defined as the change in transmittance over the transition temperature $T_t$, is a measure
of the energy-saving efficieny and depends on doping (107,108), the materials microstructure (80,95,109-111)
and the film thickness (80,88) (!!!!!Also Blackman \cite[p.~4569]{Blackman2009} talk about the dependency on
thickness of VO2 films!!!! SO include this ref!!). 
The film thickness is the most important factor of the above parameters. However, because increasing 
it would reduce the transmittance of visual light, this parameter must be carefully tuned.
%
\textbf{???Important? Film thickness:}\\
For VO$_2$ the ideal film thickness regarding visual transmission
and switching efficieny is be between 40-90nm. Here the maximal
thickness should be set correspondingly to the minimum acceptable optical transmission
\cite[p.~358]{Kamalisarvestani2013} 
\cite[p.~4569]{Blackman2009}.
%And the largest changes in optical properties are seen above a critical thickness of 70 nm (14). 
The ideal film would have the maximum acceptable thickness which corresponds to the minimum 
acceptable optical transmission. In our case this is in the range 70–90 nm thick. Also
the film must be adherent and visually appealing.''

\subsection{Tungsten doping of VO$_2$ coatings} \label{sec:vo2}
The most efficient dopant of VO$_2$ coatings is the chemical element tungsten W. 
Tungsten is also known as wolfram and can be found in the periodic table
under atomic number 74. When used as a dopant in VO$_2$ it reduces the transition temperature $T_t$
of the MST and can optimally lower $T_t$ down to about $25^{\circ}$C at 2 atom \% loading
\cite[p.~4566]{Blackman2009}. 
%
There are also other problems regarding VO$_2$ coatings affecting their applicability other than $T_t$: 
vanadium dioxide have an unappealing yellow or brown color. However, this is also improved using
tungsten as a dopant and give the coating a greener/bluer color. This is depending on the relative
amount of tungsten and increases its aesthetical properties \cite[p.~4565,4569]{Blackman2009}.
%\\
%\\
%\cite[p.~4565]{Blackman2009}\\
%p.4565\\
%In addition to the practical complications of VO$_2$, it also has an unattractive color of
%yellow/brown. Luckily, the incorporation of tungsten give it a more appealing grenn/blue color,
%depending on the relative amount of tungsten.\\
%''These tech-nology advances were not transferable into commercially relevant pro-ducts 
%because of inappropriate transition temperatures, low visible light transmission, 
%unattractive visible colours and limitedfilm durability.''\\

%\cite[p.~4566]{Blackman2009}\\
%''Tungsten has been shown to be the most effective dopant ion in reducing the MST of VO2, it can optimally lower the Tc to about 25 °C at 2 atom\% loading.''
%my own words $\rightarrow$: The most effective dopant in reducing the transition temperature of 
%vanadium dioxide has shown to be tungsten, which can lower $T_t$ to about $25^{\circ}$C at
%2 atom \% loading.
%\\
%\\
%\cite[p.~4569]{Blackman2009}\\
%The film should be as thick as it can without breaking the minimum acceptable opticle transmission.
%(In the case of Blackman's study, this range were within a thickness of 70-90nm)
%In addition the film must be adherent (sticking to the surface) and visually appealing.
%\\
%\\
%p.4569 (skrevet om av meg)\\
%Using APCVD metod (reliable and reproducible) to deposit the glazing on the glass,
%they managed, by changing the ratio of vanadium precursor to water, to change the
%film apperance from powdery brown to adherent and transparent, with an improvement to greener color.
%This is accompanied by a change in film transmission in the near IR and the size of the MST 
%is dependent on the thickness of the film being above a certain level.
%Controllable doping of the films with tungsten was demonstrated. This allows control of $T_t$
%in addition to give it an attractive blue color in transmission (from tungsten), further improving the aesthetic properties.
%\\
%\\
%Seems like the figures of this paper has done experiment with VO2 using 150nm,800nm and 30nm films.
%\\
%\\
%Also, blackman mentions how the earlier work with VO2 thin films lacked control of film thickness 
%which is crucial for optimizing the optical transmission. This resulted in that even though 
%they had the required switching temperature, the excessive thickness made the 
%optical transmission too low and therefore unsuitable for application in architectural glazing. 

