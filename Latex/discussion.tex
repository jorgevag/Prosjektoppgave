\section{\textbf{Discussion}}
Based on the litterature, the optical properties of vanadium dioxide should change drastically 
around the transition temperature $68^{\circ}$C, where it should change from a semi-conductor behavior,
being less reflective, to a more reflective metallic state.
%In Section \ref{sec:freqDepEpsilon}, it was discussed how the imaginary part of the dielectric 
%function is related to the dissipated energy of the system. Based on this, one can see that there
%is a large dissipation of energy just above the visible region. 
In Section \ref{sec:freqDepEpsilon}, some of the reasons behind the frequency dependence of the
dielectric function were discussed. Figure \ref{fig:DF} shows increasing activity in the high energy region,
at about 3.7eV, just above the visible range. There is also some activity in the low energy region below 
1.0 eV, starting at around $60^{\circ}$C. This seems to fit with the resulting resonances seen in the 
differential reflectivity, shown in Figure \ref{fig:dR}.
%one would expect increased absorbtion or reflection at these regions due to the resonant behavior. T
Furthermore, notice that the response for p-polarized is larger than for s-polarized light. 
This makes sense as p-polarized light can excite both parallel and perpencidular modes Eq.\eqref{fresCoeffP},
while the response due to s-polarized light is only due to parallel modes Eq.\eqref{fresCoeffS}.
\\
\\
%These modes should be related to the polarizabilities,
As argued in the simplified discussion, it is expected that the modes should show some dependence 
on the surface polarizability. The surface polarizability and susceptibility in Figure 
\ref{fig:alphaGamma}, \ref{fig:alphaBeta} shows a strikingly simiar behavior and only seem to differ 
in magnitude. One might be tempted to conclude a relation accoaring to 
Eq.\eqref{susceptibilityPolarizability}, with $\varepsilon_+ = 1$. 
However, even though these simulations were carried out using the same island density $\rho$,
Figure \ref{fig:gammaBetaAlpha} shows that the relation is far more complicated than stated in
the theoretical discussion. Still, for low temperatures the ratio seems to be approximately constant, at
at least for $\alpha_{\parallel}$, and of the order $\sim 0.1$.
Compared to the island density $\rho = 1/L^2 \approx 0.0005$ this is also an indication that
there are some effects not covered in the simplified theoretical introduction.
\textbf{???Should I say this? Can I say this???}
\\
\\
Looking at the peak in the IR region, Figure \ref{fig:dRlowE} and \ref{fig:IRpeak}, the temperature
dependence seem to follow the same behavior for both $r=10$nm and $r=15$nm, except that the 
latter has a larger peak amplitude of roughly a factor of 4.





\begin{itemize}
   \item Check transmission for consistensy. Check if this seems valid/physical or else i have a problem.
   \item Discuss the resonance in the reflectance and compare it to the interpolated dielectric function.
      This will allow to include the theory of the frequency dependent dielectric function.
   \item Discuss the resonance in terms of the polarizabilities/surface susceptibilities. Compare 
      the resonances.
   \item Talk about the behaviour of the reflectance. Does this make sense with what we already know
      about VO$_2$? Does this support VO$_2$ as a good smart window candidate?
   \item Also drag in the color, since this is also related to the smart window. 
      Maybe also comment on (if i've mentioned it earlier) that I didn't get the yellow/green ish color.
      only blue. But, I've also only tried thicknesses of around 10-15nm radiuses.
\end{itemize}
