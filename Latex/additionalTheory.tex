\section{Additional Theory}

\subsection{Complex permittivity and refractive index (From wikipedia) don't use, LACKING REFERENCES!}
Complex electric permittivity:
\begin{align}
   \hat{\varepsilon}_r(\omega) = \frac{\hat{\varepsilon} (\omega)}{\varepsilon_0}
\end{align}
where 
\begin{align}
   \hat{\varepsilon}_r(\omega) &= \varepsilon_r (\omega) + i\tilde{\varepsilon}_r (\omega) \\
                               &= \varepsilon_r (\omega) + i\frac{\sigma}{\omega\varepsilon_0} 
\end{align}
The complex refractive index $\hat{n}$ is given by 
\begin{align}
   \hat{n} = \sqrt{\hat{\varepsilon}_r},
\end{align}
when the magnetic properties are neglected ($\mu_r = 1$). 
From this, an expression for the complex refractive index $\hat{n} = n - \boldsymbol{i}\kappa$ can be found:
\begin{align}
   \hat{\varepsilon}_r &= \hat{n}^2 \\
   \varepsilon_r + \boldsymbol{i}\tilde{\varepsilon}_r &= (n + \boldsymbol{i} \kappa)^2 \\
   \varepsilon_r + \boldsymbol{i}\tilde{\varepsilon}_r &= n^2 - \kappa^2 + \boldsymbol{i}2n\kappa
\end{align}
giving
\begin{align}
   \varepsilon_r &= n^2 - \kappa^2     &\tilde{\varepsilon}_r  &= 2n\kappa.
\end{align}
Taking the absolute value or modulus of the relative permettivity
\begin{align}
   |\hat{\varepsilon}_r| &= \sqrt{ \varepsilon_r^2 + \tilde{\varepsilon}_r^2} \\
   |\hat{\varepsilon}_r| &= \sqrt{ (n^2 - \kappa^2)^2 + (2n\kappa)^2} \\
   |\hat{\varepsilon}_r|^2 &= (n^4 - 2n^2\kappa^2 + \kappa^4) + 4n^2\kappa^2 \\
   |\hat{\varepsilon}_r|^2 &= n^4 + 2n^2\kappa^2 + \kappa^4 \\
   |\hat{\varepsilon}_r|^2 &= (n^2 + \kappa^2)^2 \\
   |\hat{\varepsilon}_r| &= n^2 + \kappa^2 
\end{align}
and adding or substracting the real part of the permittivity, gives
\begin{align}
   |\hat{\varepsilon}_r| + \varepsilon_r &= (n^2 + \kappa^2) + (n^2 - \kappa^2) = 2n^2\\
   |\hat{\varepsilon}_r| - \varepsilon_r &= (n^2 + \kappa^2) - (n^2 - \kappa^2) = 2\kappa^2.
\end{align}
Refomulating the expression gives the real and imaginary parts of $\hat{n}$
\begin{align}
   n      &= \sqrt{ \frac{|\hat{\varepsilon}_r| + \varepsilon_r}{2}} 
           = \sqrt{ \frac{|\hat{\varepsilon}| + \varepsilon}{2\varepsilon_0}}\\
   \kappa &= \sqrt{ \frac{|\hat{\varepsilon}_r| - \varepsilon_r}{2}} 
           = \sqrt{ \frac{|\hat{\varepsilon}| - \varepsilon}{2\varepsilon_0}}
\end{align}


\subsection{Polarizability. Don't use, LACKING REFERENCES}
When a neutral atom is placed in an electric field $\boldsymbol{E}$, the field tries to rip the
atom apart by pushing the nucleus in the direction of the field and the electrons in the opposite direction.
Because of the attraction between the positive and negative charge within the atom, an equilibrium displacement
of the electrons compared to the nucleus is achieved, leaving the atom polarized and giving it a
dipole moment. The dipole moment can be approximated by
\begin{align}
   \boldsymbol{p} = \alpha \boldsymbol{E},
\end{align}
where $\alpha$ is the atomic polarizability and may depend on the detailed structure of the atom.
For more complicated situations, like an asymmetrical molecule, the gained dipole moment of the 
molecule does not necessarily have to be in the same direction as the applied electric field.
In such a case, the scalar polarizability in the expression above is replaced by a polarizability tensor
\begin{align}
   \boldsymbol{\alpha} = 
\begin{bmatrix}
   \alpha_{xx}   &   \alpha_{xy}  &  \alpha_{xz}  \\
   \alpha_{yx}   &   \alpha_{yy}  &  \alpha_{yz}  \\
   \alpha_{zx}   &   \alpha_{zy}  &  \alpha_{zz} 
\end{bmatrix}
.
\end{align}
In this way, an applied eletric field induces many dipole moments in a material. In addition,
any polar molecules will be subject to a torque, aligning it to the direction of the field.
These two mechanisms leads to the polarization $\boldsymbol{P}$ of the material
\begin{align}
   \boldsymbol{P} = \text{dipole moment per unit volume} = \varepsilon_0 \chi_e \boldsymbol{E}.
\end{align}
In the above expression, there has been assumed a linear dielectric media, where $\chi_e$ is the electric 
susceptibility and depends on the microscopic structure of the material, in addition to the external 
temperature (\cite{Griffiths},p.160-166, 179)

\subsection{The electric potential, Laplace's equation and the Uniqueness Theorem}
The usual task of electrostatics is to ompute the electric field $\boldsymbol{E}$ given a 
stationary charge distribution $\rho{\boldsymbol{r}}$
\begin{align}
   \boldsymbol{E}(\boldsymbol{r}) 
   &= \frac{1}{4 \pi \varepsilon_0} \int \frac{\boldsymbol{\hat{d}}(\boldsymbol{r'})}
                                              {d(\boldsymbol{r'})^2} 
                                                            \rho(\boldsymbol{r'}) d\!\boldsymbol{r'} 
                                                            \\
   &= \frac{1}{4 \pi \varepsilon_0} \int \frac{\boldsymbol{r} - \boldsymbol{r'}}
                                              {\big|\boldsymbol{r} - \boldsymbol{r'}\big|^3} 
                                                           \rho(\boldsymbol{r'}) d\!\boldsymbol{r'}.
\end{align}
However, it is usually simpler to calculate the potential
\begin{align}
   \label{electricPotential}
   V(\boldsymbol{r}) 
   &= \frac{1}{4 \pi \varepsilon_0} \int \frac{1}{\big|\boldsymbol{r} - \boldsymbol{r'}\big|} 
                                                           \rho(\boldsymbol{r'}) d\!\boldsymbol{r'}
\end{align}
first and then calculate the electric field from
\begin{align}
   \boldsymbol{E} = - \nabla V.
\end{align}
This might in some situations, where do do not necessarily know $\rho$ but only the total amount of 
charge, also be to tough to handle analytically. In situations like these it is better to use
Poisson's equation
\begin{align}
   \label{poisson}
   \nabla^2 V= - \frac{1}{\varepsilon_o} \rho,
\end{align}
which together with appropriate boundary conditions, is equivalent to Eq.\eqref{electricPotential}.
Very often, we are interested in finding the potential containing no charge (because the charge is 
located on the outside of our region of interest. In such cases Eq. \eqref{poisson} reduces to
Laplace's equation (\cite{Griffiths}, p.110-111)
\begin{align}
   \label{laplace}
   \nabla^2 V = 0.
\end{align}
According to the \textit{Uniqueness Theorems}, the solution to Laplace's equation is uniquely 
determined in some volume if the potential is specified on the boundary of the volume. This
easily extends to Poisson's equation by further requiring, in addition to the 
potential on the boundary, that the charge distribution throughout the region is known.
\\
\\
When considering conductors, charge are allowed to move freely and  might start to rearrange themselves,
leading to the \textit{Second uniqueness theorem}, which states that the potential in a given volume,
surrounded by conductors is uniquely determined if the total charge on each conductor is given.
\\
\\
The uniqueness theorem grants an enlarged mathematical freedom in the approach of finding the potential
of a region of space. This is because the boundary uniquely determines the potential in the region enclosed
region and any approach giving the correct boundary conditions would give you the correct potential 
function through Laplace's equation Eq. \eqref{laplace}. This allows the use of tricks, like for example
the classical \textit{method of images} (\cite{Griffiths}, p.116-121).
%SOME EXTRA STUFF:
%\textbf{D'Alembert Operator $\square$}
%\begin{align*}
  %\square  &= \partial ^{\mu} \partial _{\mu}
   %\\
           %&= \frac{1}{c^2} \frac{\partial ^2}{\partial t^2} 
               %- \frac{\partial ^2}{\partial x^2} 
               %- \frac{\partial ^2}{\partial y^2} 
               %- \frac{\partial ^2}{\partial z^2} 
   %\\
           %&= \frac{1}{c^2} \frac{\partial ^2}{\partial t^2} - \nabla ^2
%\end{align*}

%\textbf{Lorentz Gauge:} \\
%For Lorentz invariance, convenient to choose the Lorenz gauge:
%\begin{align*}
   %\square \vec{A} = \Bigg[ \frac{1}{c^2} \frac{\partial ^2}{\partial t^2} - \nabla ^2 \Bigg] \vec{A} = \mu_0 \vec{J}
%\end{align*}
%\begin{align*}
   %\square \phi = \Bigg[ \frac{1}{c^2} \frac{\partial ^2}{\partial t^2} - \nabla ^2 \Bigg] \phi = \frac{\rho}{\epsilon _0}
%\end{align*}
