\section{Additional Theory}

\subsection{The Drude model \cite[p.~?]{Jensen1985}}
%\textbf{Alt I dette kapittelet er direkte sitat foreløping:}\\
%The classical Drude model for the complex dielectric constant of a semi-conductor can be used to
%extract the mobility and the free-carrier density $n_e$ from an analysis of the reflectivity and transmittance
%data in the far infrared (1-4)=
      %\cite{Palik1979, Holm1977,Perkowitz1971,Perkowitz1974,Fan1967}.

%\hl{The Drude theory} gives the free-carrier contribution to $\varepsilon_1$ and $\varepsilon_2$ in terms
%of the \hl{plasma frequency} $\bar\omega_p$ and the \hl{electron scattering time} $\tau$ as
%\begin{align}
   %\varepsilon_1 &= \varepsilon_{\infty} \frac{1 - \bar\omega_p^2}{\omega^2 \eta}\\
   %\label{eps1Drude}
%\end{align}
%\begin{align}
   %\varepsilon_2 &= \frac{\omega_p^2}{\omega^2 \eta} \frac{ 1 }{ \omega \tau}
%\end{align}
%where $\varepsilon_{\infty}$ is the \hl{high-frequency lattice dielectric constant}. 
%\\
%\\
%p.171:\\ 
%\hl{In the far infrared,} for photon energies small compared with $k_0 T$ ($k_0$ boltzmanns const.)
%and with the energy $\hbar \omega_Q$ of the phonon involved in the scatteering,\hl{ the
%quantum result reduces to the $\lambda^2$ dependenve given by the Drude Theory, and the quasi classical
%Boltzmann transport equation (1-3)}. \hl{The departurs from the Drude theory
%at high frequencies} are associated mainly with $k$ rather than $n$, and hence,
%the transmission is affected more than the reflectivity. The latter depends on
%$k$ in the region of the reflectivity minimum, where $n \simeq 1$, but is determined 
%essentially by $n$ over the region of the absorption spectrum for which $n > k$,
%which is the region where departures from the \hl{Drude theory} would occur.
%\\
%(...)
%\\
%The response of electrons to a driving field may be followed from the quasi-classical
%limit of small $\omega$ to the quantum limit that occurs when $\hbar \omega$ is no longer small
%compared with characteristic energies of the system. In this case, a generalized Boltzmann equation
%is obtained that reduces to the quasi-classical Boltzmann transmport equation when the electron wave 
%vector $q$ tends to zero and $\omega$ is small (14-17)=
%\cite{Jensen1975,Price1966,Argyres1961,Kohn1958}
%. When $\omega$ is appreciable, one obtaines,
%under certain conditions, \hl{a solution of the Boltzmann equation in terms of a frequency-dependent
%relaxation time}. \hl{This relaxation rate}, which has been tabulated as a function of frequency and 
%carrier concentration for various materials (18-20)=
%\cite{Jensen1977,Jensen1979,Jensen1981}
%, \hl{can be used in the usual expression of the 
%classical Drude theory to obtain the quantum result}. In particular, the low-
%frequency $\hbar\omega \simeq k_0T$ limit gives a good estimate for the dc mobility as a function
%of carrier concentration. At high frequencies, in lightly doped materials in which
%polar scattering dominates, $n\alpha$ is proportional to $\lambda^3$ and $\varepsilon_2$ and $k$
%are proportional to $\lambda^4$ rather than $\lambda^3$. The real part of the dielectric constant
%is given approximately by \hl{the Drude-theory} expression and $n \simeq \sqrt{\varepsilon_{\infty}}$
%for $\bar\omega_p \ll \omega \ll G/\hbar$, where $G/\hbar$ is the frequency of the fundamental absorption
%edge and $G$ is the direct-band-gap energy of the semiconductor. As $\omega$ approaches $G/\hbar$ there
%is a small quantum-mechanical correction to $\varepsilon_1$ and hence to $n$.
%A summary of the results of the quantum theory is given in Section II.
%\\
%\\
%p.176:\\
%\hl{Comparison with experimental data}\\
%A calculation of $\varepsilon_1$ appropriate to electrons in polar semiconductors with the
%band structure of the Kane theory and based on the quantum density-matrix equation of motion yields 
%\hl{a high frequency modification to Eq.}\eqref{eps1Drude} . 
%For $\omega \tau \gg 1$ and $X < 0.1$, ($X = \hbar \omega/G$) one obtains (18,22)
%\begin{align}
   %\varepsilon_1 &= \Bigg[ \frac{\varepsilon_{\infty}}{1 - X} \Bigg]\big[ 1 - (X/\varepsilon_{\infty}) - \bar\omega_p^2/\omega^2 \big] \\
                %&= \Bigg[ \frac{\varepsilon_{\infty}}{1 - X} \Bigg]\big[ 1 - \bar\omega_p^2/\omega^2 \big] \\
                %&= \varepsilon_{\infty}(1 - \bar\omega_p^2/\omega^2), \:\:\:\:\: X \ll 1
%\end{align}
%We note that $1/\varepsilon_{\infty} <\sim$ 0.1 and hence $X/\varepsilon \ll 1$, for compounds we 
%consider, and this term can be neglected. \hl{In the limit $X \ll 1$, the quasi-classical high-frequency
%Drude result is recovered, as required}. For $X \sim 0.1$, there is a high-frequency correction given by
%Eq.(some equation), which is used to calculate the numerical values of $\varepsilon_1 = n^2 - k^2$.
%The major modification of the classical result is dispersion in $n$ as one approaches the fundamental
%absorption edge \cite{Jensen1983}.


\begin{thebibliography}{9}

      %Main article for dielectric function vs refractive index:
      \bibitem{Jensen1985}
       Jensen B.
       The quantum extension of the Drude-Zener theory in polar Semiconductors.
       Handbook of optical constants of Solids, Five-Volume 1997 (1985??)(9)<-it's chapter 9;169-170

       %''Drude model references''
      %\bibitem{Palik1979}
       %Palik ED, Hold RT.
       %Nondestructire evaluation of semiconductor materials and devices.
       %Plenum 1979 (Chap. 7) 
      %\bibitem{Holm1977}
         %Holm RT, Gibson, Palik ED, 
         %J. Appl.Phys. 48,212 (1977)
      %\bibitem{Perkowitz1971}
         %Perkowitz S. J.Phys.Chem.Solids 32, 2267 (1971)
      %\bibitem{Perkowitz1974}
         %Perkowitz S, Thorland RH, Phys.Rev. B9, 545 (1974)
      %\bibitem{Fan1967}
         %Fan HY.
         %Semiconductors and semimetals.
         %(R.K. Willardson and A.C. Beer eds.),
         %Vol.3, Academic Press, New York 1067

      %\bibitem{Jensen1975}
         %Jensen B. Ann. Phys. 95,229 (1975).
      %\bibitem{Price1966}
         %Price PJ. IBM J. Res. Dev. 10,395(1966)
      %\bibitem{Argyres1961}
         %Argyres PN, J. Phys.Chem.Solids 19, 66 (1961)
      %\bibitem{Kohn1958}
         %Kohn W, Luttinger JM.
         %Phys. Rev. 108, 590 (1957)
         %Kohn W, Luttinger JM.
         %Phys.Rev.109,1892(1958)

      %\bibitem{Jensen1977}
         %Jensen B. Phys. StatusSolidi. 86, 291 (1978);
         %Jensen B. SolidState commun. 24, 853 (1977)
      %\bibitem{Jensen1979}
         %Jensen B. J. Appl Phys. 50, 5800 (1979)
      %\bibitem{Jensen1981}
         %Jensen B.
         %Laser Induced damage in optical materials; 1980 (H.E. Bennet, A.J. Glass, A.H. Guenther,
         %and B.D.Newman, eds.), p.416, National bureau of standards special
         %publication 620, Boulder, Colorado, 1981.

      %\bibitem{Jensen1983}
         %Jensen B, IEEE J. 
         %Quantum Electron. QE-18, 1361 (1982); 
         %%
         %Jensen B, Torabi A, IEEE J.Quantum Electron. QE-19, 448-457,877-882,1362-1365 (1983);
         %%
         %Jensen N, Torabi A, J.Appl.Phys. 54,2030-2035,3623-3625,5945-5949 (1983).
\end{thebibliography}

\subsection{Complex permittivity and refractive index (From wikipedia) don't use, LACKING REFERENCES!}
Complex electric permittivity:
\begin{align}
   \hat{\varepsilon}_r(\omega) = \frac{\hat{\varepsilon} (\omega)}{\varepsilon_0}
\end{align}
where 
\begin{align}
   \hat{\varepsilon}_r(\omega) &= \varepsilon_r (\omega) + i\tilde{\varepsilon}_r (\omega) \\
                               &= \varepsilon_r (\omega) + i\frac{\sigma}{\omega\varepsilon_0} 
\end{align}
The complex refractive index $\hat{n}$ is given by 
\begin{align}
   \hat{n} = \sqrt{\hat{\varepsilon}_r},
\end{align}
when the magnetic properties are neglected ($\mu_r = 1$). 
From this, an expression for the complex refractive index $\hat{n} = n - \boldsymbol{i}\kappa$ can be found:
\begin{align}
   \hat{\varepsilon}_r &= \hat{n}^2 \\
   \varepsilon_r + \boldsymbol{i}\tilde{\varepsilon}_r &= (n + \boldsymbol{i} \kappa)^2 \\
   \varepsilon_r + \boldsymbol{i}\tilde{\varepsilon}_r &= n^2 - \kappa^2 + \boldsymbol{i}2n\kappa
\end{align}
giving
\begin{align}
   \varepsilon_r &= n^2 - \kappa^2     &\tilde{\varepsilon}_r  &= 2n\kappa.
\end{align}
Taking the absolute value or modulus of the relative permettivity
\begin{align}
   |\hat{\varepsilon}_r| &= \sqrt{ \varepsilon_r^2 + \tilde{\varepsilon}_r^2} \\
   |\hat{\varepsilon}_r| &= \sqrt{ (n^2 - \kappa^2)^2 + (2n\kappa)^2} \\
   |\hat{\varepsilon}_r|^2 &= (n^4 - 2n^2\kappa^2 + \kappa^4) + 4n^2\kappa^2 \\
   |\hat{\varepsilon}_r|^2 &= n^4 + 2n^2\kappa^2 + \kappa^4 \\
   |\hat{\varepsilon}_r|^2 &= (n^2 + \kappa^2)^2 \\
   |\hat{\varepsilon}_r| &= n^2 + \kappa^2 
\end{align}
and adding or substracting the real part of the permittivity, gives
\begin{align}
   |\hat{\varepsilon}_r| + \varepsilon_r &= (n^2 + \kappa^2) + (n^2 - \kappa^2) = 2n^2\\
   |\hat{\varepsilon}_r| - \varepsilon_r &= (n^2 + \kappa^2) - (n^2 - \kappa^2) = 2\kappa^2.
\end{align}
Refomulating the expression gives the real and imaginary parts of $\hat{n}$
\begin{align}
   n      &= \sqrt{ \frac{|\hat{\varepsilon}_r| + \varepsilon_r}{2}} 
           = \sqrt{ \frac{|\hat{\varepsilon}| + \varepsilon}{2\varepsilon_0}}\\
   \kappa &= \sqrt{ \frac{|\hat{\varepsilon}_r| - \varepsilon_r}{2}} 
           = \sqrt{ \frac{|\hat{\varepsilon}| - \varepsilon}{2\varepsilon_0}}
\end{align}


%SOME EXTRA STUFF:
%\textbf{D'Alembert Operator $\square$}
%\begin{align*}
  %\square  &= \partial ^{\mu} \partial _{\mu}
   %\\
           %&= \frac{1}{c^2} \frac{\partial ^2}{\partial t^2} 
               %- \frac{\partial ^2}{\partial x^2} 
               %- \frac{\partial ^2}{\partial y^2} 
               %- \frac{\partial ^2}{\partial z^2} 
   %\\
           %&= \frac{1}{c^2} \frac{\partial ^2}{\partial t^2} - \nabla ^2
%\end{align*}

%\textbf{Lorentz Gauge:} \\
%For Lorentz invariance, convenient to choose the Lorenz gauge:
%\begin{align*}
   %\square \vec{A} = \Bigg[ \frac{1}{c^2} \frac{\partial ^2}{\partial t^2} - \nabla ^2 \Bigg] \vec{A} = \mu_0 \vec{J}
%\end{align*}
%\begin{align*}
   %\square \phi = \Bigg[ \frac{1}{c^2} \frac{\partial ^2}{\partial t^2} - \nabla ^2 \Bigg] \phi = \frac{\rho}{\epsilon _0}
%\end{align*}
