\section{Introduction}
In the light of global warming, techlonogies that lower the overall energy consumptions, and thereby
decrease
energy-related carbon dioxide emmision, are more important today than in the past. 
%A significant amount of the total energy consumption in developed contries are due to buildings,
%consuming 30-40\% of the contry's total energy usage. 
Today the energy consumption of buildings in the developed contries constitute about 30-40\%
of their total energy usage and in humid regions, this increases to roughly 30\% to 50\% \cite{AlRabghi2001}\cite{Wilde2004}\cite{Kwak2010}. 
In 2010, 41\% of the primary energy
of the U.S. (being the second largest energy consumer globally), accounting for 7\% of the global energy use,
were consumed by the building sector.
This resulted in approximately 40\% of the total energy-related carbon dioxide emmision in the US. 
For comparison, the building sector in China accounted for 18\% of the CO$2$ emmision of the country, 
whereas worldwide the building energy consumption is behind 8\% of the total emmision.
\cite{buildingsEnergyDatabook}\cite{Hong2009}.
This motivates measures to be taken to reduce the building energy consumption, in order to reduce the 
related CO$_2$ emmisions.
\\
\\
%Heating, ventilation and airconditioning (HVAC) to maintain thermal comfort,
%together with lighting, made out about 60\% of the building energy consumption in 2010 
%\textbf{(???in the U.S. or in general???)}.
%\cite{buildingsEnergyDatabook}. HVAC compensates for heat loss through walls and windows,
%thermal radiation from the sun, etc.. in other words, it actively maintains 
%a comfortable indoor climate. 
Heating, ventilation and airconditioning (HVAC) help to maintain a comfortable indoor climate in buildings by
compensating for heat loss through their envelope (walls, roof, windows or any element separating
the indoor from the outdoor) or heating due to the thermal radiation from the sun.
Together with lighting, they were responsible for about 60\% of the total building energy consumption 
in 2010
\textbf{(???in the U.S. or in general???)}.
\cite{buildingsEnergyDatabook}. 
Considering thermal loss through the building 
envelope, the window elements are in fact the most energy 
%
%inefficient component \cite{Baetens2010}. Improving the thermal properties of the window
%will therefore be crucial in order to reduce the electricity costs regarding the devices used
%to maintain thermal comfort. 
inefficient components \cite{Baetens2010} and improving their thermal properties will be crucial in order
to reduce the electricity costs.
%
The thermal properties of the window depend mainly on
the outdoor contidtions, shading, building orientation and type, in addition to the
area of the window, its glass properties and glazing characteristics (13). In window standards,
the latter is the most important, because the glazing characteristics includes thermal transmittance
and ?solar parameters? \textit{Not sure if I understand the meaning of solar parameters here}(14).
\textbf{omformulere dette? sjekke hva det egentlig betyr?
\cite{Kamalisarvestani2013}, s.354 avsnitt 2 }\\
One way of improving the thermal efficieny of the window is to add some additional mechanism,
allowing the window to change its properties to the environment. An example of such improved 
windows are called ''smart-'' or ''intelligent windows'' and will be discussed in the next section.
\\
\\
\textbf{EXTRA:} Due to lighting, a window should be able to let through visible light (12).
\\
\\
\begin{itemize}
\item Two approaches to increase energy efficieny (7-10)
   \begin{itemize}
      \item Active stratergies: improving HVAC systems and building lighting.
      \item Passive stratergies: improving the thermal properties of the building envelope (elementss
         separating the indoor from outdoor), i.e. thermal insulation to wall, cool coatings on roofs
                  and coated window glazings.
   \end{itemize}
\end{itemize}
\textbf{Smart Windows}

Smart windows (or intelligent windows) are defined as a type of window that partially blocks the
solar radiation in hot wather and transmitts the solar radiation in cold weather by changing its
thermal and radiative properties dynamically(?trenger jeg ''dynamically''? er ikke dette bare
smør på flesk: changing+dynamically?) (25). The change in its optical properties can be
obtained by adding a controllable absorbing layer on the surface of the glass (26).
(The switchable reflective device (or dynamic tintable window)) The windows with the 
switchable layer can be categorized into active and passive systems. The active switchable glazing 
systems require an external triggering mechanism and offers supplementary options compared to 
passive systems. However, due to their dependency on a power supply and additional 
electronical curcuits makes them not as attractive as their passive counterparts.  
The passive devices do not require an external energy source, but switches automatically
subject to environmental change. Examples of such devices are: photochomic windows reacting to 
light and thermochromic windows, which change in accordance to the temperature (11).\\
\\
yyyyyyyyyyyyyyyyyyyyyyyyyyyyyyyyyyyyyyyyyyyyyyyyyyyyyyyyyyyyyyyyyyyyyyyyyyyyyyyy\\
Maybe give example of active system, talk about lighting and then introduce Figure 1 and 
conclude that the TCW is the best low-priced alternative.(p.356).\\
yyyyyyyyyyyyyyyyyyyyyyyyyyyyyyyyyyyyyyyyyyyyyyyyyyyyyyyyyyyyyyyyyyyyyyyyyyyyyyyy


(The information for this section was gathered by \cite{Kamalisarvestani2013}) \textbf{FJERN DETTE eller 
INKLUDER DETTE PÅ EN BEDRE MÅTE (om det er verdt å nevne) }

\cite{buildingsEnergyDatabook}
\begin{thebibliography}{9}


      %Main article, for now..., for now...
      \bibitem{Kamalisarvestani2013}
      Kamalisarvestani M, Saidur R, Mekhilef S, Javadi FS.
      \emph{Performance, materials and coathing technologies of thermochromic thin films on smart windows}, 
      PressOrSomething?, 
      Renewable and Sustainable Energy Reviews ??
      2013; 26:353-364 ??
      \textbf{ER DETTE RIGKTIG?}

      \bibitem{buildingsEnergyDatabook}
      \emph{DoE U. Buildings energy databook}
      Energy Effucuebct \& Renewable Energy Department 2011.
      \textbf{MÅ SJEKKES!}

      \bibitem{AlRabghi2001}
      Al-Rabghi OM, Hittle DC.  
      \emph{Energy simulation in buildings: overview and BLAST example.} 
      Energy Conversion and Management 
      2001;42(13):1623-35 
      \textbf{MÅ SJEKKES!}

      \bibitem{Wilde2004}
      Wilde PD, Voorden MVD.C.  
      \emph{Providing computational support for the selection of energy saving building components.} 
      Energy and Buildings 
      2004;36(8):749-58

      \bibitem{Kwak2010}
      Kwak SY, Yoo SH, Kwak SJ.
      \emph{Valuing energy-saving measures in residential buildings: a choice experiment study.}
      Energy Policy
      2010; 38(1):673-7

      \bibitem{Hong2009}
      Hong T. 
      \emph{A close look at the China design standard for energy efficiency of public buildings.}
      Energy and Buildings
      2009;41(4):426-35

      \bibitem{Baetens2010}
      Baetens R, Jelle BP, Gustavsen A.
      \emph{Properties, requirements and possibilities of smart windows for dynamic daylight and solar energy control in buldings: a state-of-the-art review}.
      Solar energy Materials and Solar Cells
      2010;94(2):87-105

\end{thebibliography}
