\section{Theory}
To understand the theoretical background behind GranFilm and scattering on diffuse surfaces, it is 
convenient to start with the simple case of scattering on a flat interface of two different half-infinite
media, see Figure !!FIGUREREF HERE!!.

The electric permittivity $\varepsilon$ and magnetic permeability $\mu$ of the media are given with subscript $1$ for the above media, and $2$ for the media below. 
Using Maxwell's equations
%
\begin{subequations}
\label{ME}
\begin{align}
   \nabla \cdot \boldsymbol{D} &= \rho \!_f           &\nabla\times\boldsymbol{E} &= - \frac{\partial \boldsymbol{B}}{\partial t} \label{ME1}\\
   \nabla \cdot \boldsymbol{B} &= 0                &\nabla \times \boldsymbol{H}&= \boldsymbol{J}\!_f + \frac{\partial \boldsymbol{D}}{\partial t}, \label{ME2}
\end{align}
\end{subequations}
%
where the electric fields, $\boldsymbol{E}$ and $\boldsymbol{D}$, and magnetic fields $\boldsymbol{B}$ and $\boldsymbol{H}$ are related through
\begin{align}
   &\boldsymbol{D} = \varepsilon \boldsymbol{E},         &\boldsymbol{H} = \frac{1}{\mu} \boldsymbol{B}
\end{align}
(assuming linear media), the fields above $\boldsymbol{E}^+(\boldsymbol{r})$ and below $\boldsymbol{E}^-(\boldsymbol{r})$ the interface can be calculated for a incident place wave 
(same goes for $\boldsymbol{B}$, $\boldsymbol{D}$ and $\boldsymbol{H}$).
So far, the boundary between the two half-infinite media has been considered to be a sharp, flat discontinuity in $\varepsilon(z)$ and $\mu (z)$. 
As soon as the surface roughness, thickness and/or impurities are taken into acount, the complexity of the problem increases.


\textbf{From ''GranFilm-Software-Article''}: \\
Information on the dieletric behavior of surfaes can often be obtained by measuring the Fresnel coefficients,
such as reflection, transmission and absorption. \\
?p.2?: for a layer with thickness negligible compared to wavelength, we introduce surface susceptibilities 
which interconnect the  fresnel coeff. and characterise the optical response of the surface. ?? DID I UNDERSTAND THIS CORRECTLY? ?? \\
Since all the Fresnel coeff. can be expressed in terms of these surface susceptibilities, the main task consist of calculating these coefficients for the appropriate geometry. \\
\textbf{Goal of GranFilm:} to calculate ?surface-susceptibility-/fresnel-coefficients? and the associated measurableFresnel quantities for various surface layer geometries. 
\begin{itemize}
\item GranFilm is free open-source software
\end{itemize}
\subsection{Mie resonances/ plasmon absorption modes}
\textbf{From ''GranFilm-Software-Article''}: \\
when small metallic particles, the resonances can be absorbed by visible light and strongly affect the fresnel 
coefficients depending on the particle morphology.
%
\subsection{Quasistatic Approximation}
%
\subsection{Electromagnetic excess fields }
%
\textbf{From ''GranFilm-Software-Article''}: \\
(Bedeaux and Vlieger) \\
Difference between the bulk extrapolated fields and the real fields. The BC at the dividing surface
(which drive all fresnel coeff.) are given in terms of the integrated excess fields perpendicular to
the surface. \\
Bedeaux and Vlieger $\rightarrow$ formalism of excess quantities (does not require exact knowledge of the near suface EM-field behaviour). \\
Excess fields are defined as the differene between the real fields and the bulk fields extrapolated to the surface. E.g. for the electric field $E(r)$ the excess quantity is defined as
\begin{align}
   E_{ex} (r) = E(r) - E^-(r) \theta(-z) - E^+(r)\theta(z),
\end{align}
where $\theta(z)$ is the Heaviside function and the superscript $\pm$ are used to indicate the region above (+) and below (-) the dividing interface at $z = 0$. 
The excess field is only significant close to the surface, since $E(r, \omega) \rightarrow \rightarrow E^{\pm}(r,\omega)$  for $z \rightarrow \pm \infty$. \\
\textit{QUESTION: \@ Is the $E^{\pm}$ field solved for a infinite homogeneous medium of type (+) and (-) respectively  OR the field in simple two media interface scattering?} \\
\textbf{From Leif Amund Lies Msc Thesis}: \\
Since the excess fields will only be significant close to the surface, they may be thought of as perturbations to the simple case of flat interface. 
\textit{OWN INTERPRETATION: \@ This meas that the bulk fields are the fields created from scattering in the interface between two half-infinite media.} \\
Instead of tediously using the quasi-static-"no source"-BC, the excess fields defined as for the electric field above are inserted into the full Maxwell equations to derive new non-sharp boundary conditions.
The result reads
%
\textbf{From Leif Amund Lies Msc Thesis AND GranFilm-Article}: \\
%
\begin{subequations}
   \label{exFieldBC} % Excess Field Boundary Conditions
\begin{align}
   \big[ \boldsymbol{E}^+_{\parallel} (\boldsymbol{r}) - \boldsymbol{E}^-_{\parallel} (\boldsymbol{r}) \big] \bigg\rvert _{z = 0} 
       &= i \omega \hat{z} \times \! \boldsymbol{M}^s_{\parallel}(\boldsymbol{r}_{\parallel}) \:-\: \nabla\!_{\parallel} P^s_{z}(\boldsymbol{r}_{\parallel}) 
       \label{exFieldBC1} \\ 
   \big[ D^+_{z} (\boldsymbol{r}) - D^-_{z} (\boldsymbol{r}) \big] \bigg\rvert _{z = 0} 
      &= - \nabla\!_{\parallel} \boldsymbol{P}^s_{\parallel}(\boldsymbol{r}_{\parallel}) 
      \label{exFieldBC2} \\ 
   \big[ \boldsymbol{H}^+_{\parallel} (\boldsymbol{r}) - \boldsymbol{H}^-_{\parallel} (\boldsymbol{r}) \big] \bigg\rvert _{z = 0} 
      &= i \omega \hat{z} \times \! \boldsymbol{P}^s_{\parallel}(\boldsymbol{r}_{\parallel}) \:-\: \nabla\!_{\parallel} M^s_{z}(\boldsymbol{r}_{\parallel})  
      \label{exFieldBC3} \\ 
   \big[ B^+_{z} (\boldsymbol{r}) - B^-_{z} (\boldsymbol{r}) \big] \bigg\rvert _{z = 0} 
      &= - \nabla\!_{\parallel} \boldsymbol{M}^s_{\parallel}(\boldsymbol{r}_{\parallel}), 
      \label{exFieldBC4}  
\end{align}
\end{subequations}
%
which is derived in Vlieger and Bedaux's \textit{Optical Properties of Surfaces} (p.21). Here the quantities with superscript $s$ are the so-called excess polarization and magnetization densities
\begin{subequations}
\label{surfQuant} %Surface Quantities
\begin{align}
   \boldsymbol{P}^s(\boldsymbol{r}\!_{\parallel}) &= \big( \boldsymbol{D}^s_{\parallel}(\boldsymbol{r}\!_{\parallel}), \:\: - \varepsilon_0 E^s_{z}(\boldsymbol{r}\!_{\parallel}) \big) \label{surfQuant1}\\
   \boldsymbol{M}^s(\boldsymbol{r}\!_{\parallel}) &= \big( \boldsymbol{B}^s_{\parallel}(\boldsymbol{r}\!_{\parallel}), \:\: - \mu_0 H^s_{z}(\boldsymbol{r}\!_{\parallel}) \big) , \label{surfQuant2}
\end{align}
\end{subequations}
and the quantities on the right hand side are the excess fields integrated along the z-axis,
\begin{subequations}
\label{intExQuant} % integrated excess quantities
\begin{align}
   \boldsymbol{D}^s_{\parallel}(\boldsymbol{r}) &= \!\!\!\!\!\!\!\!\! \int\limits ^{\:\:\:\:\:\:\:\:\:\:+\infty}_{\!\!\!\!\!\!\!\!\!\!\!\!\!\!\!-\infty} \!\!\!\!\!\!\!\!\! d\!z\: \boldsymbol{D}\!_{ex,\parallel}(\boldsymbol{r}),
   &E^s_{z}(\boldsymbol{r}) = \!\!\!\!\!\!\!\!\! \int\limits ^{\:\:\:\:\:\:\:\:\:\:+\infty}_{\!\!\!\!\!\!\!\!\!\!\!\!\!\!\!-\infty} \!\!\!\!\!\!\!\!\! d\!z\: E\!_{ex,z}(\boldsymbol{r}) \label{intExQuant1}\\
   \boldsymbol{B}^s_{\parallel}(\boldsymbol{r}) &= \!\!\!\!\!\!\!\!\! \int\limits ^{\:\:\:\:\:\:\:\:\:\:+\infty}_{\!\!\!\!\!\!\!\!\!\!\!\!\!\!\!-\infty} \!\!\!\!\!\!\!\!\! d\!z\: \boldsymbol{B}\!_{ex,\parallel}(\boldsymbol{r}),
   &H^s_{z}(\boldsymbol{r}) = \!\!\!\!\!\!\!\!\! \int\limits ^{\:\:\:\:\:\:\:\:\:\:+\infty}_{\!\!\!\!\!\!\!\!\!\!\!\!\!\!\!-\infty} \!\!\!\!\!\!\!\!\! d\!z\: H\!_{ex,z}(\boldsymbol{r}). \label{intExQuant2}
\end{align}
\end{subequations}
\textit{OWN INTERPRETATION: These integrated excess quantities are equivalent of representing the the excess fields in a single Dirac term $\delta (z)$ located at the surface ($z = 0$), e.g. such that the electric field can be 
written as}
%
\begin{align}
   E(r) =  E^-(r) \theta(-z) + E^s(r)\delta (z) +  E^+(r)\theta(z).
\end{align}
%
\textit{OWN INTERPRETATION: Demanding that this fullfuls Maxwell's Equations, one obtains the Equations in \eqref{exFieldBC} }.
\textit{OWN INTERPRETATION: The simplest way to link the Surface polarization and magnetization density to the extrapolated bulk fields(?Sigma indexed fields?)involves a
symmetric constitutive tensor $\xi^s_e(\omega)$ (ref B,V-OPoS).}
\begin{align}
   \boldsymbol{P}^s(\boldsymbol{r}\!_{\parallel}) = \xi ^s_e \: \big[ \boldsymbol{E}_{\parallel, \Sigma}(\boldsymbol{r}\!_{\parallel}), \:\: - D\!_{z, \Sigma}(\boldsymbol{r}\!_{\parallel}) \big]
\end{align}
\textit{OWN INTERPRETATION: The above relation is restricted to non-magnetic materials, i.e. that $\boldsymbol{M}^s(\boldsymbol{r}\!_{\parallel}) = 0$.   The $\Sigma$ index denotes the arithmetic mean of the upper and lower
bulk fields, e.g. $ \boldsymbol{E}_{\parallel, \Sigma} = \big\{ \boldsymbol{E}^+_{\parallel} \!( \boldsymbol{r}\!_{\parallel} ) +  \boldsymbol{E}^-_{\parallel} \! (\boldsymbol{r}\!_{\parallel}) \big\} \big/2 $.
If the interface are z = 0 is isotropic and symmetric, the interfacial tensor $\xi ^s_e$ is diagonal}:
\begin{align}
\xi ^s_e = 
\begin{bmatrix}
   \gamma   &   0       &  0      \\
   0        &   \gamma  &  0      \\
   0        &   0       &  \beta 
\end{bmatrix}
.
\end{align}
Here the coefficients $\gamma$ and $\beta$ are called the (first-order) surface susceptibilities (or here, constitutive coefficients). The constitutive coefficients of second order, $\delta$ and $\tau$ describe
a non-local dependence (??SPATIAL VARIATIONS IN THE EXCESS QUANTITIES??) \\
\textbf{Fresnel Coefficients} \\
...need to write some more here...
\begin{subequations}
   \label{fresCoeffS}
\begin{align}
   r_s(\omega) &= \frac{n\!_{_-} \cos \theta_i - n\!_{_+} \cos \theta_t + i(\omega/c) \gamma}{n\!_{_-} \cos \theta_i + n\!_{_+} \cos \theta_t - i(\omega/c) \gamma} \label{fresCoeffS1} \\
   t_s(\omega) &= \frac{2 n\!_{_-} \cos \theta_i}{n\!_{_-} \cos \theta_i + n\!_{_+} \cos \theta_t - i(\omega/c) \gamma} \label{fresCoeffS2}
\end{align}
\end{subequations}
%
\begin{subequations}
\label{fresCoeffP}
\begin{align}
   r_p(\omega) &= \frac{\kappa\!_{_-}(\omega) -i(\omega / c) \gamma \cos \theta_i \cos \theta_t + i(\omega/c)n\!_{_-} n\!_{_+} \varepsilon\!_{_-}\beta\sin^2 \theta_i }
   {\kappa\!_{_+}(\omega) -i(\omega / c) \gamma \cos \theta_i \cos \theta_t - i(\omega/c) n\!_{_-} n\!_{_+} \varepsilon\!_{_-} \beta \sin^2 \theta_i }, \label{fresCoeffS1}\\
   t_p(\omega) &= \frac{2n\!_{_-} \cos \theta_i \big[ 1 + (\omega/2c)^2 \varepsilon\!_{_-} \gamma \beta \sin ^2 \theta_i \big]}
   {\kappa\!_{_+}(\omega) -i(\omega / c) \gamma \cos \theta_i \cos \theta_t - i(\omega/c) n\!_{_-} n\!_{_+} \varepsilon\!_{_-} \beta \sin^2 \theta_i }, \label{fresCoeffS2}\\
   \kappa\!_{\pm} &= \big[ n\!_{_+} \cos \theta _i \pm n\!_{_-} \cos \theta_t  \big]\Bigg[ 1 - \frac{\omega^2}{4c^2} \varepsilon\!_{_-} \gamma \beta \sin ^2 \theta_i \Bigg]. \label{fresCoeffS3}
\end{align}
\end{subequations}
%
%\begin{subequations}
%\begin{align}
   %r_p(\omega) &= \frac{\kappa _-(\omega) -i(\omega / c) \gamma \cos \theta_i \cos \theta_t + i(\omega/c)n_-n_+\varepsilon_-\beta\sin^2 \theta_i }
               %{\kappa _+(\omega) -i(\omega / c) \gamma \cos \theta_i \cos \theta_t - i(\omega/c)n_-n_+\varepsilon_-\beta\sin^2 \theta_i } \\
   %t_p(\omega) &= \frac{2n_- \cos \theta_i \big[ 1 + (\omega/2c)^2 \varepsilon_- \gamma \beta \sin ^2 \theta_i \big]}
               %{\kappa _+(\omega) -i(\omega / c) \gamma \cos \theta_i \cos \theta_t - i(\omega/c)n_-n_+\varepsilon_-\beta\sin^2 \theta_i } \\
   %\kappa_{\pm} &= \big[ n_+ \cos \theta _i \pm n_- \cos \theta_t  \big]\Bigg[ 1 - \frac{\omega^2}{4c^2} \varepsilon_- \gamma \beta \sin ^2 \theta_i \Bigg]
%\end{align}
%\end{subequations}
From Eq. \eqref{fresCoeff1}-Eq.\eqref{fresCoeff3} we can observe that $p$-polarized excite the surface in both parallel and perpendicular direction, relative to the surface. 






\subsection{.}


\textbf{D'Alembert Operator $\square$}
\begin{align*}
  \square  &= \partial ^{\mu} \partial _{\mu}
   \\
           &= \frac{1}{c^2} \frac{\partial ^2}{\partial t^2} 
               - \frac{\partial ^2}{\partial x^2} 
               - \frac{\partial ^2}{\partial y^2} 
               - \frac{\partial ^2}{\partial z^2} 
   \\
           &= \frac{1}{c^2} \frac{\partial ^2}{\partial t^2} - \nabla ^2
\end{align*}

\textbf{Lorentz Gauge:} \\
For Lorentz invariance, convenient to choose the Lorenz gauge:
\begin{align*}
   \square \vec{A} = \Bigg[ \frac{1}{c^2} \frac{\partial ^2}{\partial t^2} - \nabla ^2 \Bigg] \vec{A} = \mu_0 \vec{J}
\end{align*}
\begin{align*}
   \square \phi = \Bigg[ \frac{1}{c^2} \frac{\partial ^2}{\partial t^2} - \nabla ^2 \Bigg] \phi = \frac{\rho}{\epsilon _0}
\end{align*}


